\uplevel{\bfseries Fill in the correct Swedish word (EngToSwe1, Lektion 3)}
\begin{flushleft}
    Also write down definitive, plural and definitive plural form
    for each word,\\
    if possible, remember:
    \begin{equation*}
        \begin{aligned}
            -a &\Rightarrow -or \Rightarrow -orna & \\
            \ldots &\Rightarrow -ar \Rightarrow -arna & \\
            \text{last syll} &\Rightarrow -er \Rightarrow -erna & \\
            \text{a,i,e,o,u} &\Rightarrow -n \Rightarrow -na & \\
            \ldots &\Rightarrow \underline{\phantom{no}} \Rightarrow -en & \\
        \end{aligned}
        \begin{aligned}
            &\left.\vphantom{\begin{aligned}
                -a &\Rightarrow -or \Rightarrow -orna & \\
                \ldots &\Rightarrow -ar \Rightarrow -arna & \\
                \text{last syll} &\Rightarrow -er \Rightarrow -erna & \\
            \end{aligned}}\right\rbrace\quad\text{en}\\
            &\left.\vphantom{\begin{aligned}
                \text{a,i,e,o,u} &\Rightarrow -n \Rightarrow -na & \\
                \ldots &\Rightarrow \underline{\phantom{no}} \Rightarrow -en & \\
            \end{aligned}}\right\rbrace\quad\text{ett}
        \end{aligned}
    \end{equation*}
\end{flushleft}
\begin{center}
    \begin{tabular}{|c c c c c c|}
        \hline
        grammatik & grammatiken & glosa & ett hus & från & cykla \\
        ett arbete & en skola & Sverige & vanligt & att & stort \\
        ett sovrum & det finns & nej & ingen & ett möte & hej \\
        hur & bara & bra & ja & en vän & nu \\
        kan & hinna & varför & så & ska & kanske \\
        tack & hej då & vill &&& \\
        \hline
    \end{tabular}
\end{center}

\begin{questions}
    \begin{multicols}{2}
        \raggedcolumns
        \question grammar \f[grammatik]
        \question vocabulary word \f[glosa]
    \end{multicols}
\end{questions}