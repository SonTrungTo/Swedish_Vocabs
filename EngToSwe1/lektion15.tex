\uplevel{\bfseries Fill in the correct Swedish word (EngToSwe1, Lektion 15)}

\begin{flushleft}
    \textbf{Compound nouns}
    \begin{itemize}
        \item en problem\textbf{hund}
        \item ett hund\textbf{problem}
    \end{itemize}

    -a, -e $\Rightarrow$ \_ or -o or +s
    \begin{itemize}
        \item ett barn + en cykel $\Rightarrow$ en barncykel
        \item en flicka + en cykel $\Rightarrow$ en flickcykel
        \item ett kyrka + en dörr $\Rightarrow$ en kyrkdörr
        \item en pojke + en sko $\Rightarrow$ en pojksko
        \item en människa + en hand $\Rightarrow$ en människ\underline{o}hand
        \item ett gymnasium + en lärare $\Rightarrow$ en gymnasi\underline{e}lärare
        \item ett blåbär + en skog $\Rightarrow$ en blåbär\underline{s}skog
        \item en kvinna + ett ansikte $\Rightarrow$ ett kvinn\underline{o}ansikte
        \item en man + ett ansikte $\Rightarrow$ ett man\underline{s}ansikte
    \end{itemize}
\end{flushleft}

\begin{center}
    \begin{tabular}{|c c c c c c|}
        \hline
        första & andra & tredje & fjärde & femte & sjätte \\
        sjunde & åttonde & nionde & tionde & elfte & tolfte \\
        trettonde & fjortonde & femtonde & sextonde & sjuttonde & artonde \\
        nittonde & tjugonde & tjugoförsta & tjugoandra & tjugotredje & tjugofjärde \\
        tjugofemte & tjugosjätte & tjugosjunde & tjugoåttonde & tjugonionde & trettionde \\
        trettioförsta & trettioandra & fyrtionde & femtionde & sextionde & sjuttionde \\
        åttionde & nittionde & hundrade &  &  &  \\
        \hline
    \end{tabular}
\end{center}

\begin{questions}
    \begin{multicols}{2}
        \raggedcolumns
        \question first \f[första]
        \question second \f[andra]
        \question third \f[tredje]
        \question fourth \f[fjärde]
        \question fifth \f[femte]
        \question sixth \f[sjätte]
        \question seventh \f[sjunde]
        \question eighth \f[åttonde]
        \question ninth \f[nionde]
        \question tenth \f[tionde]
        \question eleventh \f[elfte]
        \question twelfth \f[tolfte]
        \question thirteenth \f[trettonde]
        \question fourteenth \f[fjortonde]
        \question fifteenth \f[femtonde]
        \question sixteenth \f[sextonde]
        \question seventeenth \f[sjuttonde]
        \question eighteenth \f[artonde]
        \question nineteenth \f[nittonde]
        \question twentieth \f[tjugonde]
        \question twenty-first \f[tjugoförsta]
        \question twenty-second \f[tjugoandra]
        \question twenty-third \f[tjugotredje]
        \question twenty-fourth \f[tjugofjärde]
        \question twenty-fifth \f[tjugofemte]
        \question twenty-sixth \f[tjugosjätte]
        \question twenty-seventh \f[tjugosjunde]
        \question twenty-eighth \f[tjugoåttonde]
        \question twenty-ninth \f[tjugonionde]
        \question thirtieth \f[trettionde]
        \question thirty-first \f[trettioförsta]
        \question thirty-second \f[trettioandra]

        \ldots

        \question fortieth \f[fyrtionde]
        \question fiftieth \f[femtionde]
        \question sixtieth \f[sextionde]
        \question seventieth \f[sjuttionde]
        \question eightieth \f[åttionde]
        \question ninetieth \f[nittionde]
        \question one hundredth \f[hundrade]
    \end{multicols}
\end{questions}
