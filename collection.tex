\documentclass[addpoints,a3paper,11pt]{exam}
\usepackage{natbib}
\usepackage{amsmath}
\usepackage[swedish]{babel}
\usepackage[affil-it]{authblk}
\usepackage[utf8]{inputenc}
\usepackage{multicol,microtype,changepage}
\newcommand{\f}[1][{}]{\fillin[#1][0.5in]}
\newcommand{\note}[1]{\textit{#1}}
\pagestyle{headandfoot}
\chead[Leadoo Marketing Technologies \\
    Swedish self-test vocabulary \\
    Son To]{}
\cfoot{Page \thepage\ of \numpages}
\rfoot{\iflastpage{THE END}{Please go on to the next page\ldots}}
\begin{document}
    \title{\bfseries Swedish Master}
    \author{Son To}
    \affil{Leadoo Marketing Technologies
    \thanks{I thank my employer!}}
    \date{25. heinäkuuta 2021}
    \maketitle

    \begin{coverpages}
        This is the review of all vocabularies that appear
        in various texts;
        namely \cite{EngToSwe1}, \cite{Riv};
        in my attempt to master Swedish, once and for all.
    \end{coverpages}

    \begin{center}
        \fbox{\fbox{\parbox{5.5in}{\centering
            This is the self-review made by me using
            the language \TeX\ written by Donald Knuth with
            \LaTeX\ as its successor in implementation. The
            documentation for this review comes from exam.cls
            made by Philip Hirschhorn.}}}
    \end{center}

    \vspace{0.1in}

    \makebox[\textwidth]{Name:\enspace\hrulefill}

    \vspace{0.2in}

    \makebox[\textwidth]{Year of birth:\enspace\hrulefill}

    \uplevel{\bfseries Fill in the correct Swedish word (EngToSwe1, pronounciation)}
\begin{center}
    \begin{tabular}{|c c c c c c|}
        \hline
        en arm & en hand & en katt & ett glas & ett par & en tomat \\
        en boll & en doktor & en klocka & ett kilo & en orm & en ost \\
        \hline
    \end{tabular}
\end{center}

\begin{questions}
    \begin{multicols}{3}
        \raggedcolumns
        \question an arm \fillin
        \question a hand \fillin
        \question a cat \fillin
        \question a ball \fillin
        \question a doctor \fillin
        \question a clock \fillin
        \question a kilogram \fillin
        \question a snake \fillin
        \question a cheese \fillin
    \end{multicols}
\end{questions}
    \uplevel{\bfseries Fill in the correct Swedish word (EngToSwe1, Lektion 1)}
\begin{center}
    \begin{tabular}{|c c c c c c|}
        \hline
        jag & du & han & hon & den & det \\
        vi & ni & de & en lektion & det här & är \\
        en familj & en pappa & i & heter & en mamma & och \\
        lektionen  & familjen & pappan & mamman & en flicka & till \\
        flickan & till & en pojke & pojken & som & en syster \\
        systern & en bror & brodern & har & också & varje \\
        en dag & daggen & går & ut &  med  &  men \\
        inte & vem & själv & vad & en intervju & intervjun \\
        \hline
    \end{tabular}
\end{center}

\begin{questions}
    \begin{multicols}{2}
        \raggedcolumns
        \question I \f[jag]
        \question you(singular) \f[du]
        \question he \f[han]
        \question she \f[hon]
        \question it (en-nouns) \f[den]
        \question it (ett-nouns) \f[det]
        \question we \f[vi]
        \question you(plural) \f[ni]
        \question they \f[de]
        \question a lesson \f[en lektion]
        \question the lession \f[lektionen]
        \question this \f[det här]
        \question am/is/are \f[är]
        \question a family \f[en familj]
        \question a father \f[en pappa]
        \question in \f[in]
        \question am/us/are called \f
        \question a mother \f[en mamma]
        \question and \f[och]
        \question the lession \f[lektionen]
        \question the family \f[familjen]
        \question the father \f[pappan]
        \question the mother \f[mamman]
        \question a girl \f[en flickan]
        \question to, of \f[till]
        \question the girl \f[flickan]
        \question a boy \f[en pojke]
        \question the boy \f[pojken]
        \question who, which, as, like \f[som]
        \question a sister \f[en syster]
        \question the sister \f[systern]
        \question a brother \f[en bror]
        \question the brother \f[brodern]
        \question have/has \f[har]
        \question also \f[också]
        \question every \f[varje]
        \question a day \f[en dag]
        \question the day \f[daggen]
        \question go \f[går]
        \question out \f[ut]
        \question with \f[med]
        \question but \f[men]
        \question not \f[inte]
        \question who \f[vem]
        \question self \f[själv]
        \question what \f[vad]
        \question an interview \f[en intervju]
        \question the interview \f[intervjun]
    \end{multicols}
\end{questions}
    \uplevel{\bfseries Fill in the correct Swedish word (EngToSwe1, Lektion 2)}
\begin{flushleft}
    Also write down infinitive vs present tense verb forms (-a vs -ar/-er).
\end{flushleft}
\begin{center}
    \begin{tabular}{|c c c c c c|}
        \hline
        komma & bo & där & ett hus & från & cykla \\
        ett arbete & en skola & Sverige & vanligt & att & stort \\
        \hline
    \end{tabular}
\end{center}

\begin{questions}
    \begin{multicols}{2}
        \raggedcolumns
        \question to come \f[komma]
        \question to live \f[bo]
    \end{multicols}
\end{questions}
    \uplevel{\bfseries Fill in the correct Swedish word (EngToSwe1, Lektion 3)}
\begin{flushleft}
    Also write down definitive, plural and definitive plural form
    for each word,\\
    if possible, remember:
    \begin{equation*}
        \begin{aligned}
            -a &\Rightarrow -or \Rightarrow -orna & \\
            \ldots &\Rightarrow -ar \Rightarrow -arna & \\
            \text{last syll} &\Rightarrow -er \Rightarrow -erna & \\
            \text{a,i,e,o,u} &\Rightarrow -n \Rightarrow -na & \\
            \ldots &\Rightarrow \underline{\phantom{no}} \Rightarrow -en & \\
        \end{aligned}
        \begin{aligned}
            &\left.\vphantom{\begin{aligned}
                -a &\Rightarrow -or \Rightarrow -orna & \\
                \ldots &\Rightarrow -ar \Rightarrow -arna & \\
                \text{last syll} &\Rightarrow -er \Rightarrow -erna & \\
            \end{aligned}}\right\rbrace\quad\text{en}\\
            &\left.\vphantom{\begin{aligned}
                \text{a,i,e,o,u} &\Rightarrow -n \Rightarrow -na & \\
                \ldots &\Rightarrow \underline{\phantom{no}} \Rightarrow -en & \\
            \end{aligned}}\right\rbrace\quad\text{ett}
        \end{aligned}
    \end{equation*}
\end{flushleft}
\begin{center}
    \begin{tabular}{|c c c c c c|}
        \hline
        grammatik & grammatiken & glosa & sur & glad & mycket \\
        typiskt & barn & alla & ute & säga & förstå \\
        trädgård & här & tycka & roligt & aha & två \\
        minut & något & utbytesstudent & extra & många & fråga \\
        om & akta & dig & lyfta & ben & fot \\
        \hline
    \end{tabular}
\end{center}

\begin{questions}
    \begin{multicols}{2}
        \raggedcolumns
        \question grammar \f[grammatik]
        \question vocabulary word \f[glosa]
        \question sulky, sour \f[sur]
        \question happy, glad \f[glad]
        \question very \f[mycket]
        \question typical \f[typiskt]
        \question child \f[barn]
        \question all \f[alla]
        \question outside \f[ute]
        \question say \f[säga]
        \question understand \f[förstå]
        \question garden \f[trädgård]
        \question here \f[här]
        \question think \f[tycka]
        \question fun \f[roligt]
        \question I see \f[aha]
        \question two \f[två]
        \question a minute \f[minut]
        \question something, anything \f[något]
        \question exchange student \f[utbytesstudent]
        \question extra \f[extra]
        \question many \f[många]
        \question ask \f[fråga]
        \question if, about \f[om]
        \question watch out, mind \f[akta]
        \question yourself, you \f[dig]
        \question to lift \f[lyfta]
        \question leg \f[ben]
        \question foot \f[fot]
    \end{multicols}
\end{questions}
    \uplevel{\bfseries Fill in the correct Swedish word (EngToSwe1, Lektion 4)}
\begin{flushleft}
    Apply adjective rules and fill in the correct word form whenever necessary, \\
    \begin{itemize}
        \item stor (no change) $\leftarrow$ \textbf{en(singular)} nouns.
        \item stor\textbf{t} $\leftarrow$ \textbf{ett(singular)} nouns.
        \item stor\textbf{a} $\leftarrow$ \textbf{plural} nouns.
        \item Den stor\textbf{a} katten / Lunströms stora katten $\leftarrow$ \textbf{SPECIFIC} individual/object.
        \item Alexander den stor\textbf{e} $\leftarrow$ \textbf{masculine} pronoun.
    \end{itemize}
\end{flushleft}
\begin{center}
    \begin{tabular}{|c c c c c c|}
        \hline
        vår & en/ett & två & tre & fyra & fem \\
        sex & sju & åtta & nio(nie) & tio(tie) & elva \\
        tolv & tretton & fjorton & femton & sexton & sjutton \\
        arton & nitton & tjugo & packa & resväska & liten \\
        små &&&&& \\
        \hline
    \end{tabular}
\end{center}

\begin{questions}
    \begin{multicols}{3}
        \raggedcolumns
        \question our \f[vår]
        \question one \f[en/ett]
        \question two \f[två]
        \question three \f[tre]
        \question four \f[fyra]
        \question five \f[fem]
        \question six \f[sex]
        \question seven \f[sju]
        \question eight \f[åtta]
        \question nine \f[nio(nie)]
        \question ten \f[tio(tie)]
        \question eleven \f[elva]
        \question twelve \f[tolv]
        \question thirteen \f[tretton]
        \question fourteen \f[fjorton]
        \question fifteen \f[femton]
        \question sixteen \f[sexton]
        \question seventeen \f[sjutton]
        \question eighteen \f[arton]
        \question $19$ \f[nitton]
        \question $20$ \f[tjugo]
        \question little(singular) \f[liten]
        \question little(plural) \f[små]
    \end{multicols}
\end{questions}
    \uplevel{\bfseries Fill in the correct Swedish word (EngToSwe1, Lektion 5)}

    \uplevel{\bfseries Fill in the correct Swedish word (EngToSwe1, Lektion 6)}

\begin{flushleft}
    By looking at the picture in the textbook,
    I write down the appropriate text without actually looking at
    the text itself.
    \begin{itemize}
        \item Han tar på sig sina skor
        \item Hon tar på sig sina skor
        \item Han tar på sig hennes skor
        \item Hon tar på sig hans skor
        \item Han tar på henne sina skor
        \item Hon tar på honom sina skor
        \item Han tar på henne hennes skor
        \item Hon tar på honom hans skor
    \end{itemize}
\end{flushleft}

\begin{center}
    \begin{tabular}{|c c c c c c|}
        \hline
        bil & äntligen & prata & stanna & framför & ur \\
        framme & hjälpa & sedan & in(i) & hall & ta av sig \\
        förvånad & göra/gör & regna & smuts & badrum & visa \\
        stänga/stänger & dörr & vänta & golv & bakom & bära/bär \\
        lägga/lägger & byrå & hänga/hänger & garderob & snygg & äta/äter \\
        bord &&&&& \\
        \hline
    \end{tabular}
\end{center}

\begin{questions}
    \begin{multicols}{2}
        \raggedcolumns
        \question car \f[bil,bilen,bilar,bilarna]
        \question to speak \f[prata,pratar]
        \question finally \f[äntligen]
        \question to stand/stop \f[stanna,stannar]
        \question in front of \f[framför]
        \question out of \f[ur]
        \question at the destination \f[framme]
        \question to help \f[hjälpa,hjälper]
        \question then \f[sedan]
        \question into \f[in(i)]
        \question hall \f[hall,hallen,hallar,hallarna]
        \question undress/take off \f[ta av sig]
        \question surprised \f[förvånad,förvånat,förvånade]
        \question to do/make \f[göra,gör]
        \question to rain \f[regna,regnar]
        \question dirt \f[smuts,smutsen]
        \question bathroom \f[badrum,badrummet,badrum,badrummen]
        \question to show \f[visa,visar]
        \question to close \f[stänga/stänger]
        \question door \f[dörr,dörren,dörrar,dörrarna]
        \question to wait \f[vänta,väntar]
        \question behind \f[bakom]
        \question floor \f[golv,golvet,golv,golven]
        \question to carry \f[bära,bär]
        \question to lay,put \f[lägga,lägger]
        \question chest of drawers \f[byrå,byrån,byråar,byråarna]
        \question to hang \f[hänga,hänger]
        \question closet,wardrobe \f[garderob,garderoben,garderober,garderoberna]
        \question good-looking \f[snygg,snyggt,snygga]
        \question to eat \f[äta,äter]
        \question table \f[bord,bordet,bord,borden]
    \end{multicols}
\end{questions}
    \uplevel{\bfseries Fill in the correct Swedish word (EngToSwe1, Lektion 7)}

\begin{flushleft}
    \textbf{Imperative form of the verb}
    \begin{itemize}
        \item att stanna $\Rightarrow$ jag stannar $\Rightarrow$ stanna!
        \item att hjälpa $\Rightarrow$ jag hjälper $\Rightarrow$ hjälp!
    \end{itemize}
\end{flushleft}

\begin{center}
    \begin{tabular}{|c c c c c c|}
        \hline
        tjugoett & tjugotvå & tjugotre & tjugofyra & tjugofem & tjugosex \\
        \hline
    \end{tabular}
\end{center}

\begin{questions}
    \begin{multicols}{2}
        \raggedcolumns
        \question twenty-one \f[tjugoett]
    \end{multicols}
\end{questions}
    \uplevel{\bfseries Fill in the correct Swedish word (EngToSwe1, Lektion 8)}

\begin{flushleft}
    \textbf{Comparison of \underline{adjectives}} \\
    \textsl{Short adjectives} (--are, --ast ($+$ --e(definite)))
    \begin{itemize}
        \item lätt $\Rightarrow$ lättare $\Rightarrow$ lättast
        \item varm $\Rightarrow$ varmare $\Rightarrow$ varmast
        \item den varmaste dagen
    \end{itemize}

    \textsl{Long and --isk adjectives} (mer/mera eller mest)
    \begin{itemize}
        \item estetisk $\Rightarrow$ mer estetisk $\Rightarrow$ mest estetisk
        \item förvånad $\Rightarrow$ mer förvånad $\Rightarrow$ mest förvånad
    \end{itemize}

    \textsl{Some irregular adjectives} (definite: --a or --e(masculine))
    \begin{itemize}
        \item bra $\Rightarrow$ bättre $\Rightarrow$ bäst
        \item dålig $\Rightarrow$ sämre $\Rightarrow$ sämst
        \item ung $\Rightarrow$ yngre $\Rightarrow$ yngst
        \item gammal $\Rightarrow$ äldre $\Rightarrow$ äldst
        \item liten $\Rightarrow$ mindre $\Rightarrow$ minst
        \item stor $\Rightarrow$ större $\Rightarrow$ störst
    \end{itemize}
    Det största huset. Den yngsta kvinnan. Den yngste mannen.

    mycket $\Rightarrow$ mer $\Rightarrow$ mest \\
    många $\Rightarrow$ fler $\Rightarrow$ flest
\end{flushleft}

\begin{center}
    \begin{tabular}{|c c c c c c|}
        \hline
        måndag &&&&& \\
        \hline
    \end{tabular}
\end{center}

\begin{questions}
    \begin{multicols}{2}
        \raggedcolumns
        \question Monday \f[måndag]
    \end{multicols}
\end{questions}
    \uplevel{\bfseries Fill in the correct Swedish word (EngToSwe1, Lektion 9)}

\begin{flushleft}
    \textbf{Positions}
    \begin{itemize}
        \item går \underline{in} $\Rightarrow$ goes inside from outside
        \item är \underline{inne} $\Rightarrow$ is inside
        \item går \underline{\underline{ut} ur} $\Rightarrow$ go outside from inside
        \item är \underline{\underline{ute} ur} $\Rightarrow$ is away from
    \end{itemize}

    \textbf{Predicate}
    \begin{itemize}
        \item Vi
        $\underbrace{
            \overbrace{\text{måste}}^{\text{finite}} 
            \overbrace{\text{gå}}^{\text{infinite}}
        }_{\text{predicate}}$
        till skolan nu.
    \end{itemize}

    \textbf{Particle verbs}
    \begin{itemize}
        \item tala (om) = talk (about) / tala \textbf{om} = tell
        \item tycka (om) = think (about) / tycka \textbf{om} = like
        \item titta (på) = to look (at) / titta \textbf{på} = watch
        \item hälsa (på) = to say hello (to) / hälsa \textbf{på} = visit
    \end{itemize}
\end{flushleft}

\begin{center}
    \begin{tabular}{|c c c c c c|}
        \hline
        skog & inne & tala/talar &  &  &  \\
        \hline
    \end{tabular}
\end{center}

\begin{questions}
    \begin{multicols}{2}
        \raggedcolumns
        \question forest \f[skog,skogen,skogar,skogarna]
        \question inside \f[inne]
        \question talk, tell \f[tala/talar]
    \end{multicols}
\end{questions}
    \uplevel{\bfseries Fill in the correct Swedish word (EngToSwe1, Lektion 10)}

\begin{flushleft}
    \textbf{Vad är klockan?}
    \begin{itemize}
        \item Klockan är tre.
        \item Klockan är fem över tre.
        \item Klockan är tio över tre.
        \item Klockan är kvart över tre.
        \item Klockan är tjugo över tre.
        \item Klockan är fem i halv fyra.
        \item Klockan är halv fyra.
        \item Klockan är fem över halv fyra.
    \end{itemize}
\end{flushleft}
    \uplevel{\bfseries Fill in the correct Swedish word (EngToSwe1, Lektion 11)}

\begin{flushleft}
    \textbf{Reverse when:}
    \begin{itemize}
        \item $\underbrace{\text{Nu}}_{\text{starting word}}$ förstår jag.
        \item $\underbrace{\text{När du förklarar }}_{\text{starting sub-clause}}$ förstår jag
        \item Förstår du vad jag säger?
    \end{itemize}
\end{flushleft}

\begin{center}
    \begin{tabular}{|c c c c c c|}
        \hline
        ett huvud &  &  &  &  &  \\
        \hline
    \end{tabular}
\end{center}

\begin{questions}
    \begin{multicols}{2}
        \raggedcolumns
        \question head \f[huvud,huvudet,huvuden,huvudena]
    \end{multicols}
\end{questions}

    \uplevel{\bfseries Fill in the correct Swedish word (EngToSwe1, Lektion 12)}

\begin{flushleft}
    \textbf{Location of \textsl{inte/aldrig}}
    \begin{itemize}
        \item V $+$ inte; Missan går inte ut nu. Missan vill inte gå ut nu.
        \item Reverse: S $+$ inte; Nu går Missan inte ut. Nu vill Missan inte gå ut.
        \item Sub-clauses: inte $+$ V; Jag vet att Missan inte går ut nu. Jag vet att Missan inte vill gå ut nu.
        \item väl, nog, kanske\ldots $+$ inte; Missan går nog inte ut nu. Missan vill nog inte gå ut nu.
    \end{itemize}
\end{flushleft}

\begin{center}
    \begin{tabular}{|c c c c c c|}
        \hline
        en gaffel & en tallrik & en kniv & en sked & en servett & djup/t/a \\
        djup tallrik & ett glas & en kopp & internationell/t/a & middag & i dag/idag \\
        problem & fixa/fixar & säng/en/ar/arna & kök & snäll/t/a & en påse \\
        kylskåp & koka/kokar & en potatis & hacka/hackar & lök & varandra \\
        laga/r & mat(en) & mexikansk/t/a & burk/en/ar/arna & fisk/en/ar/arna & öppna/r \\
        fy & surströmming & lukta/r & illa & nog & baka/r \\
        bröd/et/\_/en & soppa/n/soppor/na & chili/n & undra/r & tro/r & rulla/r \\
        dukad & person(en)(er)(erna) & olik/t/a & sallad/en/er/erna & god/tt/a & smaka(r) \\
        efteråt &  &  &  &  &  \\
        \hline
    \end{tabular}
\end{center}

\begin{questions}
    \begin{multicols}{2}
        \raggedcolumns
        \question fork \f[gaffel,gaffeln,gafflar,gafflarna]
        \question plate \f[tallrik,tallriken,tallrikar,tallrikarna]
        \question knife \f[kniv,kniven,knivar,knivarna]
        \question spoon \f[sked,skeden,skedar,skedarna]
        \question napkin \f[servett,servetten,servetter,servetterna]
        \question deep \f[djup/djupt/djupa]
        \question bowl \f[djup tallrik]
        \question glass \f[glas,glaset,glas,glasen]
        \question cup \f[kopp,koppen,koppar,kopparna]
        \question international \f[internationell/t/a]
        \question today \f[i dag/idag]
        \question dinner, noon \f[middag,middagen,middagar,middagarna]
        \question problem \f[problem,problemet,problem,problemen]
        \question fix/arrange \f[fixa/fixar]
        \question bed \f[säng/en/er/erna]
        \question kitchen \f[kök(et)()(en)]
        \question kind, please \f[snäll/t/a]
        \question bag \f[påse,påsen,påsar,påsarna]
        \question refrigerator \f[kylskåp,kylskåpet,kylskåp,kylskåpen]
        \question boil \f[koka,kokar]
        \question potato \f[potatis,potatisen,potatisar,potatisarna]
        \question chop \f[hacka/hackar]
        \question onion \f[lök,löken,lökar,lökarna]
        \question each other \f[varandra]
        \question prepare \f[laga,lagar]
        \question food \f[mat,maten]
        \question Mexican \f[mexikansk/t/a]
        \question can \f[burk/en/ar/arna]
        \question fish \f[fisk/en/ar/arna]
        \question open \f[öppna/r]
        \question ugh, phew \f[fy]
        \question sour herring \f[surströmming/en/ar/arna]
        \question smell \f[lukta/r]
        \question bad \f[illa]
        \question probably \f[nog]
        \question bake \f[baka/r]
        \question bread \f[bröd/et/\_/en]
        \question soup \f[soppa/n/soppor/sopporna]
        \question chili pepper \f[chili/n]
        \question wonder \f[undra/r]
        \question believe, think \f[tro/r]
        \question roll \f[rulla/r]
        \question served(table) \f[duka(d)(t)(de)]
        \question person \f[person(en)(er)(erna)]
        \question different \f[olik/t/a]
        \question salad \f[sallad(en)(er)(erna)]
        \question good \f[god,gott,goda]
        \question taste \f[smaka(r)]
        \question afterwards \f[efteråt]
    \end{multicols}
\end{questions}

    \uplevel{\bfseries Fill in the correct Swedish word (EngToSwe1, Lektion 13)}

\begin{flushleft}
    \textbf{Time of the years}
    \begin{itemize}
        \item ett år
        \item fyra årstider
        \item tolv månader
    \end{itemize}

    \textbf{Questions}
    \begin{itemize}
        \item Hur, när, var, varför, vem, vilken $+$ reverse main clause
        \begin{itemize}
            \item Hur reser du till Stockholm? $\Rightarrow$ Jag tar tåget.
        \end{itemize}
        \item Tar du tåget till Stockholm? $\Rightarrow$ $\left\{
            \begin{array}{l}
                \text{Ja, det gör jag.} \\
                \text{Nej, det gör jag inte.}
            \end{array}
        \right.$
        \item väl
        \begin{itemize}
            \item Du tar väl tåget till Stockholm? $\Rightarrow$ $\left\{
                \begin{array}{l}
                    \text{Ja, det gör jag.} \\
                    \text{Nej, det gör jag inte.}
                \end{array}
            \right.$
            \item Du tar väl \textbf{inte} tåget till Stockholm? $\Rightarrow$ $\left\{
                \begin{array}{l}
                    \text{\textbf{Jo}, det gör jag.} \\
                    \text{Nej, det gör jag inte.}
                \end{array}
            \right.$
        \end{itemize}
    \end{itemize}
\end{flushleft}

\begin{center}
    \begin{tabular}{|c c c c c c|}
        \hline
        januari & februari & mars & april & maj & juni \\
        juli & augusti & september & oktober & november & december \\
        vinter & vår & sommar & höst & årstid & månad \\
        slut &  &  &  &  &  \\
        \hline
    \end{tabular}
\end{center}

\begin{questions}
    \begin{multicols}{2}
        \raggedcolumns
        \question January \f[januari]
        \question February \f[]
        \question March \f[]
        \question April \f[]
        \question Maj \f[]
        \question June \f[]
        \question July \f[]
        \question August \f[]
        \question September \f[]
        \question October \f[]
        \question November \f[]
        \question December \f[]
        \question winter \f[vinter,vintern,vintrar,vintrarna]
        \question spring \f[vår(en)(ar)(arna)]
        \question summer \f[sommar,sommaren,somrar,somrarna]
        \question fall \f[höst(en)(ar)(arna)]
        \question season \f[årstid(en)(er)(erna)]
        \question month \f[månad(en)(er)(erna)]
        \question end \f[slut/et/\_/en]
    \end{multicols}
\end{questions}

    \uplevel{\bfseries Fill in the correct Swedish word (EngToSwe1, Lektion 14)}

\begin{flushleft}
    \textbf{Present participle}
    \begin{itemize}
        \item -nde $\Leftarrow$ V-a
        \item -ende $\Leftarrow$ andra
    \end{itemize}

    Examples,
    \begin{itemize}
        \item att cykla $\Rightarrow$ cyklande
        \item att resa $\Rightarrow$ resande
        \item att gå $\Rightarrow$ gående
        \item att se $\Rightarrow$ seende
    \end{itemize}

    Cases,
    \begin{enumerate}
        \item Lisa är blind men Olof är \textbf{seende}. (adj)
        \item Bo kommer \textbf{gående} mellan träden. (adv)
        \item Med allt mitt \textbf{resande} sitter jag mycket på tåget. (an activity)
        \item På tåget finns många \textbf{resande}. (performer of an activity)
    \end{enumerate}
\end{flushleft}

\begin{center}
    \begin{tabular}{|c c c c c c|}
        \hline
        advokat & artist & bagare & frisör & författare & ingenjör \\
        konstnär & läkare & präst & sjuksköterska & veterinär & lussekatt \\
        enda & Tyskland & Italien & ge(r) & början & gärna \\
        hemma & knappast & emot & lite & hos & både \\
        deg & recept & mobil & jäst(en) & socker,sockret & mjölk/mjölken \\
        smör(et) & mjöl(et) & saffran & ägg & russin & sak \\
        bild & runt & bit & tå(n/r/rna) & förstås & svår(t/a) \\
        universitet & sjunga/er & under & städa(r) & medan & grädda(r) \\
        \hline
    \end{tabular}
\end{center}

\begin{questions}
    \begin{multicols}{2}
        \raggedcolumns
        \question lawyer \f[advokat,advokaten,advokater,advokaterna]
        \question stage artiste \f[artist(en)(er)(erna)]
        \question baker \f[bagare/n/\_/na]
        \question hair dresser \f[frisör(en)(er)(erna)]
        \question author \f[författare/n/\_/na]
        \question engineer \f[ingenjör(en)(er)(erna)]
        \question artist \f[konstnär(en)(er)(erna)]
        \question doctor \f[läkare/n/\_/na]
        \question priest \f[präste(n)(r)(rna)]
        \question nurse \f[sjuksköterska(n)/or/orna]
        \question veterinarian \f[veterinär(en)(er)(erna)]
        \question Santa Lucia roll \f[lussekatt(en)(er)(erna)]
        \question only \f[enda]
        \question Germany \f[Tyskland]
        \question Italy \f[Italien]
        \question to give/offer \f[ge,ger]
        \question beginning \f[början]
        \question gladly \f[gärna,hellre,helst]
        \question at home \f[hemma]
        \question hardly \f[knappast]
        \question against \f[emot]
        \question some \f[lite]
        \question at, with \f[hos]
        \question both \f[både]
        \question dough \f[deg,degen,degar,degarna]
        \question recipe,prescription \f[recept(et)(\_)(en)]
        \question cell telephone \f[mobil(en)(er)(erna)]
        \question yeast \f[jäst(en)]
        \question sugar \f[socker,sockret]
        \question milk \f[mjölk/mjölken]
        \question butter \f[smör(et)]
        \question flour \f[mjöl,mjölet]
        \question saffron \f[saffran(en)/(et)]
        \question egg \f[ägg,ägget,ägg,äggen]
        \question raisin \f[russin(et)(\_)(en)]
        \question thing \f[sak,saken,saker,sakerna]
        \question picture \f[bild(en)(er)(erna)]
        \question around \f[runt]
        \question piece \f[bit,biten,bitar,bitarna]
        \question toe \f[tå(n/r/rna)]
        \question of course \f[förstås]
        \question difficult \f[svår(t/a)]
        \question university \f[universitet(et/\_/en)]
        \question to sing \f[sjunga/er]
        \question under \f[under]
        \question to clean up \f[städa(r)]
        \question while \f[medan]
        \question bake \f[grädda(r)]
    \end{multicols}
\end{questions}

    \uplevel{\bfseries Fill in the correct Swedish word (EngToSwe1, Lektion 15)}

\begin{flushleft}
    \textbf{Compound nouns}
    \begin{itemize}
        \item en problem\textbf{hund}
        \item ett hund\textbf{problem}
    \end{itemize}

    -a, -e $\Rightarrow$ \_ or -o or +s
    \begin{itemize}
        \item ett barn + en cykel $\Rightarrow$ en barncykel
        \item en flicka + en cykel $\Rightarrow$ en flickcykel
        \item ett kyrka + en dörr $\Rightarrow$ en kyrkdörr
        \item en pojke + en sko $\Rightarrow$ en pojksko
        \item en människa + en hand $\Rightarrow$ en människ\underline{o}hand
        \item ett gymnasium + en lärare $\Rightarrow$ en gymnasi\underline{e}lärare
        \item ett blåbär + en skog $\Rightarrow$ en blåbär\underline{s}skog
        \item en kvinna + ett ansikte $\Rightarrow$ ett kvinn\underline{o}ansikte
        \item en man + ett ansikte $\Rightarrow$ ett man\underline{s}ansikte
    \end{itemize}
\end{flushleft}

\begin{center}
    \begin{tabular}{|c c c c c c|}
        \hline
        första & andra & tredje & fjärde & femte & sjätte \\
        sjunde & åttonde & nionde & tionde & elfte & tolfte \\
        trettonde & fjortonde & femtonde & sextonde & sjuttonde & artonde \\
        nittonde & tjugonde & tjugoförsta & tjugoandra & tjugotredje & tjugofjärde \\
        tjugofemte & tjugosjätte & tjugosjunde & tjugoåttonde & tjugonionde & trettionde \\
        trettioförsta & trettioandra & fyrtionde & femtionde & sextionde & sjuttionde \\
        åttionde & nittionde & hundrade &  &  &  \\
        \hline
    \end{tabular}
\end{center}

\begin{questions}
    \begin{multicols}{2}
        \raggedcolumns
        \question first \f[första]
        \question second \f[andra]
        \question third \f[tredje]
        \question fourth \f[fjärde]
        \question fifth \f[femte]
        \question sixth \f[sjätte]
        \question seventh \f[sjunde]
        \question eighth \f[åttonde]
        \question ninth \f[nionde]
        \question tenth \f[tionde]
        \question eleventh \f[elfte]
        \question twelfth \f[tolfte]
        \question thirteenth \f[trettonde]
        \question fourteenth \f[fjortonde]
        \question fifteenth \f[femtonde]
        \question sixteenth \f[sextonde]
        \question seventeenth \f[sjuttonde]
        \question eighteenth \f[artonde]
        \question nineteenth \f[nittonde]
        \question twentieth \f[tjugonde]
        \question twenty-first \f[tjugoförsta]
        \question twenty-second \f[tjugoandra]
        \question twenty-third \f[tjugotredje]
        \question twenty-fourth \f[tjugofjärde]
        \question twenty-fifth \f[tjugofemte]
        \question twenty-sixth \f[tjugosjätte]
        \question twenty-seventh \f[tjugosjunde]
        \question twenty-eighth \f[tjugoåttonde]
        \question twenty-ninth \f[tjugonionde]
        \question thirtieth \f[trettionde]
        \question thirty-first \f[trettioförsta]
        \question thirty-second \f[trettioandra]

        \ldots

        \question fortieth \f[fyrtionde]
        \question fiftieth \f[femtionde]
        \question sixtieth \f[sextionde]
        \question seventieth \f[sjuttionde]
        \question eightieth \f[åttionde]
        \question ninetieth \f[nittionde]
        \question one hundredth \f[hundrade]
    \end{multicols}
\end{questions}


    \uplevel{\bfseries Fill in the correct Swedish word (RivStart1, Kapitel 1)}

\begin{flushleft}
    \textbf{Klassrumsfraser}
    \begin{itemize}
        \item Hur uttalar man \ldots? (How do you pronounce \ldots?)
        \item Hur stavar man \ldots? (How do you spell \ldots?)
        \item Kan du säga det en gång till? (Can you say it again?)
        \item Stäng boken. (Close the book.)
        \item Läs texten på sidan X. (Read the text on page X)
        \item Lyssna. (Listen.)
        \item Vad kul. (Great! How fun!)
    \end{itemize}
\end{flushleft}

\begin{center}
    \begin{tabular}{|c c c c c c|}
        \hline
        rivstart & kapitel & sida & kunna & kombinera & foto \\
        hamburgare & kaffe & kanelbulle & vatten & heta & varifrån \\
        komma & vad\ldots för & talade/talat & så klart & var/varit & var \\
        ligga & pratade/pratat & pyttelite & lyssna & dialog & vilken \\
        ha & betoning & ringa in & igen & markera & uttala \\
        ruta & nedanför & ett system & en position & se & läsa \\
        frågeord & mingla & en grupp & land & ett pronomen & sa/sade,sagt \\
        arbeta & tandläkare & busschaufför & webbdesigner & en fotograf & en servitör \\
        en kock & nähä & gjorde/gjort & plugga & studera & en design \\
        en pensionär & jobba & söka & ett jobb & lycka till & en negation \\
        en mening & ett yrke & ett IT-företag & ekonomi & programmerare & en ekonom \\
        förskolelärare & en programmering & läsa & studera till & höra(hör/hörde/hört) & kryssa för \\
        rätt(a) & ett alternativ & en biolog & en kemist & gift & en sambo \\
        singel & min(mitt,mina) & man(nen) & Schweizare & son(en/söner/sönerna) & en dotter \\
        svensk(t/a) & flickvän(nen) & bo & bo ihop & toppen & skilja(er) \\
        en källa & sedan & skild(t/da) & separerad(t/e) & tjej(en/er/erna) & pojkvän \\
        en kille & en fru & en pojke & en flicka & bonusbarn & generell \\
        stryka & nära & sortera & matematik & lätt & form \\
        stava & annan & använda & forska & institut &  \\
        \hline
    \end{tabular}
\end{center}

\begin{questions}
    \begin{multicols}{2}
        \raggedcolumns
        \question flying start \f[rivstart(en/er/erna)]
        \question chapter \f[kapitel(kapitlet/\_/kapitlen)]
        \question page \f[sid(an/or/orna)]
        \question know; can \f[kan,kunde,kunnat]
        \question combine \f[kombinera(r/de/t)]
        \question photo \f[foto(t/n/na)]
        \question hamburger \f[hamburgar(en/\_/na)]
        \question coffee \f[kaffe(et)]
        \question cinnamon roll/bun \f[kanelbull(en/ar/arna)]
        \question water \f[vatten,vattnet]
        \question be called \f[heter,hette,hetat]
        \question from where \f[varifrån]
        \question come \f[kommer,kom,kommit]
        \question what \f[vad \ldots för]
        \question spoke/spoken \f[talade/talat]
        \question of course \f[så klart]
        \question was/were/been \f[var/varit]
        \question where \f[var]
        \question lie, be situated \f[ligger,låg,legat]
        \question talked \f[pratade,pratat]
        \question a tiny bit \f[pyttelite]
        \question listen \f[lyssna(r/de/t)]
        \question dialogue \f[dialog(en/er/erna)]
        \question which \f[vilket,vilka]
        \question have/had \f[ha(r/de/ft)]
        \question emphasis \f[betoning(en/ar/arna)]
        \question circle \f[ringa in]
        \question again \f[igen]
        \question to mark \f[markera(r/de/t)]
        \question to pronounce \f[uttala(r/de/t)]
        \question box \f[rut(an/or/orna)]
        \question below \f[nedanför]
        \question system \f[system(et/\_/en)]
        \question position \f[position(en/er/erna)]
        \question see/saw/seen \f[se(r/såg/sett)]
        \question read \f[läsa(er/te/t)]
        \question question word \f[frågeord]
        \question to mingle \f[mingla(r/de/t)]
        \question group \f[grupp(en/er/erna)]
        \question country \f[land(et/länder/länderna)]
        \question pronoun \f[pronomen(et/\_/en)]
        \question said \f[sa/sade,sagt]
        \question to work \f[arbeta(r/de/t)]
        \question dentist \f[tandläkare]
        \question bus driver \f[busschaufför(en/er/erna)]
        \question webdesigner \f[webbdesigner]
        \question a photographer \f[fotograf(en/er/erna)]
        \question waiter/waitress \f[servitör(en/er/erna)]
        \question chef \f[kock(en/ar/arna)]
        \question I see \f[nähä]
        \question did/made \f[gjorde/gjort]
        \question study \f[plugga(r/de/t)]
        \question study \f[studera(r/de/t)]
        \question design \f[design(en)]
        \question pensioner \f[pensionär(en/er/erna)]
        \question to work \f[jobba(r/de/t)]
        \question to search/look for \f[söka(er/te/t)]
        \question job \f[jobb(et/\_/en)]
        \question good luck \f[lycka till]
        \question negation \f[negation(en/er/erna)]
        \question sentence \f[mening(en/ar/arna)]
        \question profession \f[yrke(t/n/na)]
        \question IT company \f[IT-företag(et/\_/en)]
        \question accounts,economics \f[ekonomi(n)]
        \question programmer \f[programmerare]
        \question accountant \f[ekonom(en/er/erna)]
        \question preschool teacher \f[förskolelärare]
        \question programming \f[programmering(en/ar/arna)]
        \question to study \f[läsa(er/te/t)]
        \question study to be \f[studera till]
        \question to hear \f[höra]
        \question to tick \f[kryssa för]
        \question right/correct \f[rätt(a)]
        \question option,answer \f[alternativ(et/\_/en)]
        \question biologist \f[biolog(en/er/erna)]
        \question chemist \f[kemist(en/er/erna)]
        \question married \f[gift(\_/a)]
        \question live-in partner \f[sambo(n/r/rna)]
        \question single \f[singel]
        \question my \f[min]
        \question husband \f[man(nen/män/männen)]
        \question Swiss \f[Schweizare]
        \question son \f[son]
        \question daughter \f[dotter(n/döttrar(na))]
        \question Swedish \f[svensk(t/a)]
        \question girlfriend \f[flickvän(ner/nerna)]
        \question live/lived \f[bo(r/dde/tt)]
        \question live together \f[bo ihop]
        \question super,great \f[toppen]
        \question to divorce \f[skilja(er/skilde/skilt)]
        \question source \f[käll(an/or/orna)]
        \question since \f[sedan]
        \question divorce \f[skild]
        \question separate \f[separerad]
        \question girlfriend \f[tjej]
        \question boyfriend \f[pojkvän]
        \question boyfriend \f[kille]
        \question wife \f[fru(n,ar,arna)]
        \question boy \f[pojke]
        \question girl \f[flicka]
        \question stepchild \f[bonusbarn]
        \question general \f[generell]
        \question underline \f[stryka]
        \question near, close to \f[nära]
        \question to sort,organize \f[sortera]
        \question mathematics \f[matematik]
        \question easy \f[lätt]
        \question form,conjugation \f[form]
        \question to spell \f[stava(r)]
        \question another \f[annan]
        \question to use \f[använda(er)]
        \question to research \f[forska(r)]
        \question institute \f[institut(et)]
    \end{multicols}
\end{questions}

    \uplevel{\bfseries Fill in the correct Swedish word (RivStart1, Kapitel 2)}

\begin{flushleft}
    \textbf{Phrases and sentences}
    \begin{itemize}
        \item Hur mår du? (How are you?)
        \item Jo tack (Thanks for asking)
        \item Hur är läget? (How is it going?)
        \item Det är lugnt. (Great, okay)
        \item Själv? (And you? How about you?)
        \item kanonbra (great).
        \item Hejsan! (Hi there!).
        \item Fint. (Great).
        \item Allt väl? (All well?)
        \item Helt okej (Completely fine).
        \item För mycket (Too much).
        \item Jodå (allright).
        \item Som vanligt (as usual).
        \item Tja!(=Tjena!) (Hey! Hi!).
        \item Hur står det till? (How are you?).
        \item Jag är förkyld. (I have a cold).
        \item västra Sverige. (Western Sweden).
    \end{itemize}
\end{flushleft}

\begin{center}
    \begin{tabular}{|c c c c c c|}
        \hline
        må & ett läge & lugn(t,a) & själv & så där & ganska \\
        faktiskt & trött & dricka & lite & hälsa & retur \\
        (in)formell  & markering & betonad & intonation & en fras & en ton \\
        upp & ner & ett slut &  &  &  \\
        \hline
    \end{tabular}
\end{center}

\begin{questions}
    \begin{multicols}{2}
        \raggedcolumns
        \question to feel \f[må(r,dde,tt)]
        \question situation \f[läge(t,n,na)]
        \question cool, calm \f[lugn(t,a)]
        \question myself,yourself \f[själv(a)]
        \question so-so \f[så där]
        \question fairly, quite \f[ganska]
        \question actually \f[faktiskt]
        \question tired \f[trött]
        \question to drink \f[dricka(er)]
        \question a bit \f[lite]
        \question to greet \f[hälsa(r)]
        \question return \f[retur]
        \question formal/informal \f[(in)formell]
        \question mark \f[markering(en,ar,arna)]
        \question emphasized \f[betonad]
        \question intonation \f[intonation]
        \question phrase \f[fras(en)]
        \question pitch \f[ton(en)]
        \question end \f[slut(et)]
    \end{multicols}
\end{questions}

    \uplevel{\bfseries Fill in the correct Swedish word (RivStart1, Kapitel 3)}

\begin{flushleft}
    \textbf{Phrases and sentences}
    \begin{itemize}
        \item När är du född? (When were you born?)
        \item Hur många? (How many?)
        \item Samma år som (Same year as)
        \item Jag fyller 83 i år (I turn 83 this year)
        \item Hur mycket? (How much?)
    \end{itemize}
\end{flushleft}

\begin{center}
    \begin{tabular}{|c c c c c c|}
        \hline
        räkna(r) & siffra(n,or,orna) & plats(en,er,erna) & ljud(et,\_,en) & bakre & främre \\
        Kina & jämn(t,a) & udda & tal(et,\_,en) & baklänges & kasta(r,de,t) \\
        tärning(en) & öva(r,de,t) & plus & minus & gata(n,or,orna) & ett nummer \\
        telefonnummer & ringa(er,de,t) & imorgon & när & född & lillasyster(n) \\
        småsystrar(na) & morfar & ohjojoj & jamen & faktiskt inte & fylla(er/de/t) \\
        ungefär &  &  &  &  &  \\
         &  &  &  &  &  \\
         &  &  &  &  &  \\
         &  &  &  &  &  \\
        \hline
    \end{tabular}
\end{center}

\begin{questions}
    \begin{multicols}{2}
        \raggedcolumns
        \question count \f[räkna(r/de/t)]
        \question digit,number \f[siffra(n)]
        \question place \f[plats]
        \question sound \f[ljud(et)]
        \question back \f[bakre]
        \question front \f[främre]
        \question China \f[Kina]
        \question even \f[jämn(t/a)]
        \question odd  \f[udda]
        \question number \f[tal(et,\_,en)]
        \question backward \f[baklänges]
        \question throw/roll \f[kasta(r)]
        \question dice \f[tärning(en,ar,arna)]
        \question practice \f[öva(r,de,t)]
        \question plus \f[plus]
        \question minus \f[minus]
        \question street \f[gata(n,or,orna)]
        \question number \f[numret,\_,numren]
        \question telephone number \f[telefonnummer]
        \question to call \f[ringa]
        \question tomorrow \f[imorgon]
        \question when \f[när]
        \question born \f[född]
        \question little sister \f[lillasyster(n)]
        \question little sisters \f[småsystrar(na)]
        \question grandfather(maternal) \f[morfar]
        \question oh,oh no \f[ohjojoj]
        \question yet, but \f[jamen]
        \question actually not \f[faktiskt inte]
        \question turn(age) \f[fylla(er/de/t)]
        \question about, approximately \f[ungefär]
    \end{multicols}
\end{questions}

    \uplevel{\bfseries Fill in the correct Swedish word (RivStart1, Kapitel 4)}

\begin{flushleft}
    \textbf{Phrases and sentences}
    \begin{itemize}
        \item en annan dag (another day)
        \item ta en fika (take a coffee break)
        \item nej tack (no thank you)
        \item Det var bra så (That is all!)
        \item Det blir \ldots\ kronor (That comes to \ldots\ kronor)
        \item Tar ni kort? (Do you take credit cards?)
        \item Vill du ätä här eller ta med? (Do you want to eat here or take away?)
        \item Var det bra så? (Will that be all?)
        \item Vill du ha \ldots\ ? (Do you want \ldots\ ?)
        \item Jag tar den. (I will take it.)
        \item Jag skulle vilja ha (I would like)
        \item Den/det där (that)
        \item Kan jag få? (Can I have?)
        \item Ja tack, nästa? (Next please?)
        \item Något mer? (Anything else?)
        \item Nej, det var bra så tack (That's all thanks.)
        \item Vad ska ni se? (What are you going to see?)
        \item Vad kostar det? (How much does it cost?)
        \item Var finns\ldots? (Where is\ldots?)
    \end{itemize}
\end{flushleft}

\begin{center}
    \begin{tabular}{|c c c c c c|}
        \hline
        måste & köpa & närbutik & kontantkort & näsduk & följa(er) med \\
        snus(et) & vilja(vill) & kebabställe(t) & gärna & kompis(en) & tyvärr \\
        farfar & te & bulle(n) & kaka(n) & smörgås(en) &  muntlig övning \\
        partner(n) & nästa & skulle vilja & kosta(r) & krona(n,or,orna) & Något annat? \\
        javisst & kort(et,\_,en) & sätta(er) in & kod(en,er,erna) & slå(r) koden & slog, slagit \\
        namnteckning & leg(et,\_,en) & kvitto(t,n,na) & meny(n,en,er) & mellanläsk(en,\_,en) & vänta(r) lite \\
        lyxbulle(n) & varsågod & strax & ovanför & tjuga(n,or,orna) & jag vill ha \\
        nedan & apelsin(en) & äpple(t) & glass(en) & prislista(n,or,orna) & dosa(n,or,orna) \\
        godispåse(n) & dammsugare(n) & morotskaka & juice(n,r,rna) & frimärke(t,n,na) & mazarin(en) \\
        några & duk(en,ar,arna) & kontant(a) & flera & sammansatta ord & snabbmatställe \\
        torg(et,\_,en) & bokstav(en) & bokstäver(na) & päron(et,\_,en) & gurka(n,or,orna) & (vin)druva(n) \\
        purjolök(en) & väldigt & semester(n) & helvete(t,n,na) & hänga(er) med & persika(an) \\
        mataffär(en) & hylla(n,or,orna) & till vänster & till höger & första & andra \\
        tredje & fjärde & bageri(et) & mejeri(et) & kött(et,\_,en) & hygien(en) \\
        fisk(en,ar) & skaldjur(et) & chark(en) & godis(et) & dryck(en,er,erna) & snacks \\
        falukorv(en) & förlåt mig & fil(en) & köttfärs(en) & bakprodukt(en) & ostbåge(n) \\
        tvål(en) & smör(et) & mjöl(et) & räka(n,or,orna) & sill(en,ar,arna) & socker(-kret) \\
        saltlakrits(et) & fläskkotlett(en) & burk(en) & förpackning(en) & duschkräm(en) & tub(en,er,erna) \\
         & & & & ask(en,ar,arna) &  \\
        & & & & &  \\
        & & & & &  \\
        & & & & &  \\
        & & & & &  \\
        & & & & &  \\
        & & & & &  \\
        & & & & &  \\
        \hline
    \end{tabular}
\end{center}

\begin{questions}
    \begin{multicols}{2}
        \raggedcolumns
        \question must \f[måste]
        \question to buy \f[köpa]
        \question local grocery store \f[närbutik]
        \question prepaid card \f[kontantkort(et,\_,en)]
        \question handkerchief \f[näsduk(en,ar,arna)]
        \question join in, come along\f[följa(er) med]
        \question snuff \f[snus(et)]
        \question want \f[vilja(vill)]
        \question kebab place \f[kebabställe(t,\_,ena)]
        \question happily, willingly \f[gärna]
        \question friend \f[kompis(en,ar,arna)]
        \question unfortunately \f[tyvärr]
        \question grandfather \f[farfar]
        \question tea \f[te]
        \question bun \f[bulle(n)]
        \question cookie,biscuit \f[kaka(n)]
        \question sandwich \f[smörgås]
        \question oral exercise \f[muntlig övning]
        \question partner \f[partner(n)]
        \question next \f[nästa]
        \question would like to \f[skulle vilja]
        \question to cost \f[kosta(r,de,t)]
        \question crown \f[krona(n)]
        \question anything else? \f[Något annat]
        \question yes of course \f[javisst]
        \question card \f[kort(et,\_,en)]
        \question put in \f[sätta(er,satte,satt) in]
        \question code \f[kod(en,er,erna)]
        \question punch the code \f[slå(r) koden]
        \question punched \f[slog, slagit]
        \question signature \f[namnteckning(en,ar,arna)]
        \question ID \f[leg(et,\_,en)]
        \question receipt \f[kvitto(t,n,na)]
        \question menu \f[meny]
        \question medium soft drink \f[mellanläsk]
        \question wait, hang on \f[vänta(r) lite]
        \question en tjugolapp \f[tjuga]
        \question luxury bun \f[lyxbulle(n,ar,arna)]
        \question you're welcome \f[varsågod]
        \question soon \f[strax]
        \question above \f[ovanför]
        \question I want \f[jag vill ha]
        \question price list \f[prislista]
        \question below \f[nedan]
        \question orange \f[apelsin]
        \question apple \f[äpple]
        \question small box \f[dosa]
        \question ice cream \f[glass(en,ar,arna)]
        \question bag of candy \f[godispåse(n,ar,arna)]
        \question postage stamp \f[frimärke(t,n,na)]
        \question vacuum cleaner \f[dammsugare(n,\_,arna)]
        \question carrot cake \f[morotskaka(n)]
        \question almond pastry \f[mazarin]
        \question juice \f[juice]
        \question few \f[några]
        \question table cloth \f[duk(en,ar,arna)]
        \question cash \f[kontant(a)]
        \question compound words \f[sammansatta ord]
        \question several \f[flera]
        \question fast food place \f[snabbmatställe(t,n,na)]
        \question square \f[torg]
        \question letter \f[bokstav(en)]
        \question (the)letters \f[bokstäver(na)]
        \question cucumber \f[gurka(n,or,orna)]
        \question pear \f[päron(et,\_,en)]
        \question grape \f[vindruva(n,or,orna)]
        \question leek \f[purjolök(en,ar,arna)]
        \question really \f[väldigt]
        \question join in \f[hänga(er) med]
        \question vacation/holiday \f[semester(n)]
        \question hell \f[helvete(t,n,na)]
        \question peach \f[persika(an)]
        \question grocery store \f[mataffär(en)]
        \question shelf \f[hylla(n,or,orna)]
        \question to the left \f[till vänster]
        \question to the right \f[till höger]
        \question first \f[första]
        \question second \f[andra]
        \question third \f[tredje]
        \question fourth \f[fjärde]
        \question bakery \f[bageri(et,er,erna)]
        \question dairy \f[mejeri(et,er,erna)]
        \question meat \f[kött(et,\_,en)]
        \question hygiene \f[hygien(en)]
        \question fish \f[fisk(en,ar,arna)]
        \question shellfish \f[skaldjur(et,\_,en)]
        \question cured meat \f[chark(en)]
        \question drink \f[dryck(en,er,erna)]
        \question candy \f[godis(et)]
        \question snacks \f[snacks]
        \question baking ingredient \f[bakprodukt(en,er,erna)]
        \question Swedish bologna sausage \f[falukorv(en)]
        \question excuse me \f[förlåt mig]
        \question sour milk \f[fil(en)]
        \question minced meat \f[köttfärs(en,er,erna)]
        \question cheese curl \f[ostbåge(n)]
        \question soap \f[tvål(en)]
        \question butter \f[smör(et)]
        \question flour \f[mjöl(et)]
        \question shrimp \f[räka(n,or,orna)]
        \question herring \f[sill(en,ar,arna)]
        \question sugar \f[socker]
        \question salt liquorice \f[saltlakrits(et)]
        \question pork chop \f[fläskkotlett(en)]
        \question body wash \f[duschkräm(en)]
        \question packaging \f[förpackning(en)]
        \question can \f[burk(en,ar,arna)]
        \question tube \f[tub(en,er,erna)]
        \question box \f[ask(en,ar,arna)]
        \question  \f[]
        \question  \f[]
        \question  \f[]
        \question  \f[]
        \question  \f[]
        \question  \f[]
        \question  \f[]
        \question  \f[]
        \question  \f[]
        \question  \f[]
    \end{multicols}
\end{questions}

    \uplevel{\bfseries Fill in the correct Swedish word (RivStart1, Kapitel 5)}

\begin{flushleft}
    \textbf{Phrases and sentences}
    \begin{itemize}
        \item Ska vi gå på \ldots? (Shall we go to \ldots?)
        \item Vilken dag? (Which day?)
        \item Var går den? (Where does it play?)
        \item Vilka spelar? (Who is playing?)
        \item Jag tycker om att \ldots (I like to \ldots)
    \end{itemize}
\end{flushleft}

\begin{center}
    \begin{tabular}{|c c c c c c|}
        \hline
        teater(n) & teatrar(na) & experiment(et,\_,en) & bio(n) & biograf(en) & lördag(en) \\
        lust(en) & hockey(n) & onsdag(en,ar,arna) & arena(n) & skoj(et) & fixa(r,de,t) \\
        fredag & låta(\_er) & biljett(en,er,erna) & lät/låtit & pjäs(en) & drömspel(et,\_,en) \\
        inte precis & favorit(en) & varför inte & i stället & ikväll & gratis \\
        poesi(n) & veckodag & akrobatisk(t,a) & torsdag & söndag & betyda(\_er) \\
        hög ton & situation(en) & typ(en,er,erna) & melodi(n) & annons(en) & fotboll(en,ar,arna)  \\
         & & & & &  \\
         & & & & &  \\
         & & & & &  \\
         & & & & &  \\
         & & & & &  \\
         & & & & &  \\
         & & & & &  \\
         & & & & &  \\
         & & & & &  \\
         & & & & &  \\
         & & & & &  \\
         & & & & &  \\
         & & & & &  \\
         & & & & &  \\
         & & & & &  \\
         & & & & &  \\
         & & & & &  \\
         & & & & &  \\
         & & & & &  \\
        \hline
    \end{tabular}
\end{center}

\begin{questions}
    \begin{multicols}{2}
        \raggedcolumns
        \question theater \f[teater(n)]
        \question experiment \f[experiment(et)]
        \question cinema \f[bio(n)]
        \question cinema \f[biograf(en,er,erna)]
        \question Saturday \f[lördag(en,ar,arna)]
        \question desire \f[lust(en,ar,arna)]
        \question hockey \f[hockey(n)]
        \question Wednesday \f[onsdag(en)]
        \question stadium \f[arena(n)]
        \question fun \f[skoj(et)]
        \question sort out, buy \f[fixa(r,de,t)]
        \question ticket \f[biljett(en,er,erna)]
        \question Friday \f[fredag]
        \question to sound \f[låta(\_er)]
        \question play \f[pjäs(en)]
        \question dream play \f[drömspel(et,\_,en)]
        \question not exactly \f[inte precis]
        \question favorite \f[favorit(en,er,erna)]
        \question instead \f[i stället]
        \question why not \f[varför inte]
        \question tonight \f[ikväll]
        \question for free \f[gratis]
        \question poetry \f[poesi(n)]
        \question acrobatics \f[akrobatisk(t,a)]
        \question Thursday \f[torsdag]
        \question Sunday \f[söndag]
        \question mean \f[betyda]
        \question high pitch \f[hög ton]
        \question situation \f[situation(en)]
        \question type \f[typ(en)]
        \question melody \f[melodi(n,er,erna)]
        \question football \f[fotboll(en)]
        \question advertisement \f[annons(en,er,erna)]
        \question  \f[]
        \question  \f[]
        \question  \f[]
        \question  \f[]
        \question  \f[]
        \question  \f[]
        \question  \f[]
        \question  \f[]
        \question  \f[]
        \question  \f[]
        \question  \f[]
        \question  \f[]
        \question  \f[]
        \question  \f[]
        \question  \f[]
        \question  \f[]
        \question  \f[]
        \question  \f[]
        \question  \f[]
        \question  \f[]
        \question  \f[]
        \question  \f[]
        \question  \f[]
        \question  \f[]
        \question  \f[]
        \question  \f[]
        \question  \f[]
        \question  \f[]
        \question  \f[]
        \question  \f[]
        \question  \f[]
        \question  \f[]
        \question  \f[]
    \end{multicols}
\end{questions}

    \uplevel{\bfseries Fill in the correct Swedish word (RivStart1, Kapitel 6)}

\begin{flushleft}
    \textbf{Phrases and sentences}
    \begin{itemize}
        \item Är det sant? (Is it true? Really?)
        \item Din då? (What about yours?)
        \item Nämen! (No, really!)
        \item Stämmer det? (Is that correct?)
        \item Lite långsammare, tack. (A bit slower, thanks.)
        \item Kan du säga det igen? (Could you that again?)
    \end{itemize}
\end{flushleft}

\begin{center}
    \begin{adjustwidth}{-2cm}{}
        \begin{tabular}{|c c c c c c|}
            \hline
            faster(n) & Nobelpristagare(n) & stadshus(et) & pris(et,er,erna) & kemi(n) & 80-talet \\
            examen & stipendi(um)(et) & forskning(en) & besöka(er/te/t) & molekyl(en) & få(r) barn \\
            brev(et,\_,en) & fick(fått) barn & i höstas & otrolig(t,a) & släkting(en) & fabror(dern) \\
            kusin(en) & mormor & farmor & moster(n) & förälder(n) & plastförälder(n) \\
            syskon(et) & bonus(en,ar,arna) & bonussyskon(et) & släktmiddag(en) & julmat(en) & prisutdelning(en) \\
            nästa gång & vandrahem(met) & leta(r/de/t) & intervjua(r/de/t) & tidsadverb & objektspronomen \\
            passa(r) in & far(dern/fäder) & mor(dern) & morbror(dern) & barnbarn & separera(r/de/t) \\
            plastpappa & låtsaspappa & regnbåge(n) & regnbågsfamilj & psykolog(en) & relativt pronomen \\
            meteorolog & ge(r) respons & dansare(n,\_,na) & anteckna(r/de/t) & gav/gett & vacker(t,vackra) \\
            viktig(t/a) & flera gånger & dynamit(en) & speciell(t,a) & dela(r/de/t) & fysik(en,er,erna) \\
            medicin(en) & litteratur(en) & konserthus(et) & riksbanken(en) & ceremoni(n) & fredspris(et) \\
            varje & prissumma(n) & miljon(en,er,erna) & fetsal(en,ar,arna) & olik(t,a) & dekorera(r/de/t) \\
            just & växa(er,te,t) & flytta(r/de/t) & experimentera(r/de/t) & växa upp & katastrof(en) \\
            dö(r/dog/t) & nitroglycerin(en) & bygga(er/de/t) & explodera(r/de/t) & patent(et,\_,en) & fabrik(en,er,erna) \\
            arbete(t,n,na) & tjäna(r/de/t) & tjäna pengar & fortsatte/fortsatt & cirka & fortsätta(er) \\
            relation(en) & starta(r/de/t) & fond(en,er,erna) & testamente(t,n,na) & resa(er/te/t) & dra(r,drog,dragit) \\
            streck(et) & byta(er) roller & understruken & alltså & kläder(na) & karriär(en,er,erna) \\
            firma(n,or) & klädfirma(n) & gissa(r/de/t) & är känd för & vuxen(t,na) & barndom(en,ar) \\
            som vuxen & sent i livet & konstruera(r) & komponera(r/de/t) & kämpa(r) för & uppfinna(er) \\
            uppfann & uppfunnit & grunda(r/de/t) & vinna(er,vann) & vunnit & konstnär(en) \\
            uppfinnare(n) & aktivist(en) & kompositör(en) & forskare(n,\_,na) &  &  \\
            \hline
        \end{tabular}
    \end{adjustwidth}
\end{center}

\begin{questions}
    \begin{multicols}{2}
        \raggedcolumns
        \question aunt(paternal) \f[faster(n)]
        \question Nobel laureate \f[Nobelpristagare(n)]
        \question city hall \f[stadshus(et)]
        \question prize \f[pris(et)]
        \question chemistry \f[kemi(n)]
        \question 80s \f[80-talet]
        \question degree \f[examen(examina)]
        \question stipend \f[stipendium]
        \question research \f[forskning(en)]
        \question to visit \f[besöka(er)]
        \question molecule \f[molekyl(en,er,erna)]
        \question have children \f[få(r) barn]
        \question got children \f[fick(fått) barn]
        \question letter \f[brev(et,\_,en)]
        \question last autumn \f[i höstas]
        \question incredible \f[otrolig(t,a)]
        \question relative \f[släkting(en,ar,arna)]
        \question uncle \f[fabror]
        \question cousin \f[kusin(en)]
        \question grandmother(maternal) \f[mormor]
        \question grandmother(paternal) \f[farmor]
        \question aunt(maternal) \f[moster(n)]
        \question parent \f[förälder(n)]
        \question step parent \f[plastförälder(n)]
        \question sibling \f[syskon(et,\_,en)]
        \question bonus, step \f[bonus(en)]
        \question step-sibling \f[bonussyskon(et)]
        \question family dinner \f[släktmiddag(en)]
        \question Christmas food \f[julmat(en)]
        \question award ceremony \f[prisutdelning(en)]
        \question next time \f[nästa gång]
        \question hostel \f[vandrahem(met)]
        \question look for,search \f[leta(r/de/t)]
        \question to interview \f[intervjua(r/de/t)]
        \question adverbs of time \f[tidsadverb(et)]
        \question object pronoun \f[objektspronomen]
        \question fit in \f[passa(r/de/t) in]
        \question father \f[far(dern)]
        \question mother \f[mor(dern)]
        \question maternal uncle \f[morbror]
        \question grandchild \f[barnbarn]
        \question to separate \f[separera(r)]
        \question stepfather \f[plastpappa(n)]
        \question stepfather \f[låtsaspappa(n)]
        \question rainbow \f[regnbåge(n)]
        \question rainbow family \f[regnbågsfamilj(en)]
        \question psychologist \f[psykolog(en)]
        \question meteorologist \f[meteorolog(en)]
        \question to respond \f[ge(r) respons]
        \question dancer \f[dansare(n,\_,na)]
        \question to take note \f[anteckna(r)]
        \question gave/given \f[gav/gett]
        \question beautiful \f[vacker(t/vackra)]
        \question important \f[viktig(t,a)]
        \question several times \f[flera gånger]
        \question dynamite \f[dynamit(en)]
        \question special \f[speciell(t,a)]
        \question share/give out \f[dela(r/de/t)]
        \question physics \f[fysik(en,er,erna)]
        \question medicine \f[medicin(en)]
        \question literature \f[litteratur(en)]
        \question concert hall \f[konserthus(et)]
        \question national bank \f[riksbanken(en)]
        \question ceremony \f[ceremoni(n)]
        \question peace prize \f[fredspris(et)]
        \question each \f[varje]
        \question prize sum \f[prissumma(n)]
        \question different \f[olik(t,a)]
        \question to decorate \f[dekorera(r)]
        \question exactly \f[just]
        \question to grow \f[växa(er/te/t)]
        \question to experiment \f[experimentera(r/de/t)]
        \question to grow up \f[växa upp]
        \question catastrophe \f[katastrof(en)]
        \question to die \f[dö(r/dog/tt)]
        \question nitroglycerin \f[nitroglycerin]
        \question to build \f[bygga(er/de/t)]
        \question to explode \f[explodera(r/de/t)]
        \question patent \f[patent(et,\_,en)]
        \question factory \f[fabrik(en,er,erna)]
        \question job \f[arbete(t,n,na)]
        \question to save \f[tjäna(r/de/t)]
        \question save money \f[tjäna pengar]
        \question to continue \f[fortsätta(er)]
        \question continued \f[fortsatte/fortsatt]
        \question approximately \f[cirka]
        \question relationship \f[relation(en,er,erna)]
        \question to start \f[starta(r/de/t)]
        \question fund \f[fond(en,er,erna)]
        \question will \f[testamente(t,n,na)]
        \question to travel \f[resa(er/te/t)]
        \question pull, draw \f[dra(r,drog,dragit)]
        \question line \f[streck(et,\_,en)]
        \question change roles \f[byta roller]
        \question underlined \f[understruken(na)]
        \question so, thus \f[alltså]
        \question clothes \f[kläder(na)]
        \question career \f[karriär(en,er,erna)]
        \question company \f[firma(n,or,orna)]
        \question clothing company \f[klädfirma(n)]
        \question to guess \f[gissa(r/de/t)]
        \question childhood \f[barndom(en,ar,arna)]
        \question is known for \f[är känd för]
        \question adult \f[vuxen(t,na)]
        \question as an adult \f[som vuxen]
        \question later in life \f[sent i livet]
        \question compose \f[komponera(r/de/t)]
        \question invent \f[uppfinna(er,uppfann,uppfunnit)]
        \question construct \f[konstruera(r/de/t)]
        \question to fight for \f[kämpa(r/de/t) för]
        \question to found \f[grunda(r/de/t)]
        \question to win \f[vinna(er,vann,vunnit)]
        \question artist \f[konstnär(en,er,erna)]
        \question composer \f[kompositör(en,er,erna)]
        \question inventor \f[uppfinnare(n,\_,na)]
        \question activist \f[aktivist(en,er,erna)]
        \question researcher \f[forskare(n,\_,na)]
    \end{multicols}
\end{questions}

    \uplevel{\bfseries Fill in the correct Swedish word (RivStart1, Kapitel 7)}

\begin{flushleft}
    \textbf{Phrases and sentences}
    \begin{itemize}
        \item 
    \end{itemize}
\end{flushleft}

\begin{center}
    \begin{adjustwidth}{-2cm}{}
        \begin{tabular}{|c c c c c c|}
            \hline
            månad(en,er,erna) & hårvax(et,\_,en) & i slutet av & outlet(en,s) & dyr(t,a) & fota(r/de/t) \\
            mode(t,n,na) & mest & second hand-affär & handla(r/de/t) & rea(n,or,orna) & strumpa(n,or,orna) \\
            hår(et,\_,en) & trumma(n) & trummor(na) & trumset(et,\_,en) & begagnad(t,e) & i våras \\
            tavla(n,or,orna) & loppis(en,ar,arna) & nätet & fungera(r/de/t) & procent(en) & textil(en,er,erna) \\
            slänga(er/de/t) & garderob(en) & indefinit & plagg(et,\_,en) & byxor(na) & jacka(n) \\
            trosor(na) & kjol(en,ar,arna) & slips(en,ar,arna) & bh & kavaj(en) & skjorta(n) \\
            linne(t,n,na) & kalsonger(na) & vit(t,a) & grön(t,a) & grå(tt,a) & orange(\_,a) \\
            randig(t,a) & rutig(t,a) & gul(t,a) & brun(t,a) & svart(a) & rosa \\
            prickig(t,a) &  &  &  &  &  \\
             &  &  &  &  &  \\
             &  &  &  &  &  \\
             &  &  &  &  &  \\
             &  &  &  &  &  \\
             &  &  &  &  &  \\
             &  &  &  &  &  \\
             &  &  &  &  &  \\
             &  &  &  &  &  \\
             &  &  &  &  &  \\
             &  &  &  &  &  \\
             &  &  &  &  &  \\
             &  &  &  &  &  \\
             &  &  &  &  &  \\
             &  &  &  &  &  \\
             &  &  &  &  &  \\
             &  &  &  &  &  \\
             &  &  &  &  &  \\
             &  &  &  &  &  \\
             &  &  &  &  &  \\
             &  &  &  &  &  \\
             &  &  &  &  &  \\
             &  &  &  &  &  \\
             &  &  &  &  &  \\
             &  &  &  &  &  \\
             &  &  &  &  &  \\
             &  &  &  &  &  \\
             &  &  &  &  &  \\
             &  &  &  &  &  \\
             &  &  &  &  &  \\
             &  &  &  &  &  \\
             &  &  &  &  &  \\
             &  &  &  &  &  \\
             &  &  &  &  &  \\
             &  &  &  &  &  \\
             &  &  &  &  &  \\
             &  &  &  &  &  \\
             &  &  &  &  &  \\
             &  &  &  &  &  \\
             &  &  &  &  &  \\
             &  &  &  &  &  \\
             &  &  &  &  &  \\
             &  &  &  &  &  \\
             &  &  &  &  &  \\
             &  &  &  &  &  \\
             &  &  &  &  &  \\
             &  &  &  &  &  \\
             &  &  &  &  &  \\
             &  &  &  &  &  \\
             &  &  &  &  &  \\
             &  &  &  &  &  \\
             &  &  &  &  &  \\
             &  &  &  &  &  \\
             &  &  &  &  &  \\
             &  &  &  &  &  \\
             &  &  &  &  &  \\
             &  &  &  &  &  \\
             &  &  &  &  &  \\
            \hline
        \end{tabular}
    \end{adjustwidth}
\end{center}

\begin{questions}
    \begin{multicols}{2}
        \raggedcolumns
        \question month \f[månad(en,er,erna)]
        \question hair wax \f[hårvax(et,\_,en)]
        \question at the end of \f[i slutet av]
        \question outlet \f[outlet(en,s)]
        \question expensive \f[dyr(t,a)]
        \question to take a photo \f[fota(r/de/t)]
        \question fashion \f[mode(t,n,na)]
        \question mostly \f[mest]
        \question second-hand shop \f[second hand-affär]
        \question to shop/buy \f[handla(r/de/t)]
        \question used, second hand \f[begagnad(t,e)]
        \question last spring \f[i våras]
        \question sale \f[rea(n,or,orna)]
        \question sock \f[strumpa(or,orna)]
        \question hair \f[hår(et,\_,en)]
        \question drum \f[trumma(n,or,orna)]
        \question drumset \f[trumset(et,\_,en)]
        \question painting \f[tavla(n,or,orna)]
        \question flea market \f[loppis(en,ar,arna)]
        \question the Internet \f[nätet]
        \question to work \f[fungera(r/de/t)]
        \question percent \f[procent(en)]
        \question textile \f[textil(en,er,erna)]
        \question throw away \f[slänga(er/de/t)]
        \question wardrobe \f[garderob(en)]
        \question indefinite \f[indefinit]
        \question piece of clothing, garment \f[plagg(et,\_,en)]
        \question trousers \f[byxor(na)]
        \question jacket \f[jacka(n,or,orna)]
        \question women's underwear \f[trosor(na)]
        \question skirt \f[kjol(en,ar,arna)]
        \question tie \f[slips(en,ar,arna)]
        \question bra \f[bh]
        \question suit \f[kvaj(en,er,erna)]
        \question shirt \f[skjorta(n,or,orna)]
        \question tank top \f[linne(t,n,na)]
        \question men's underwear \f[kalsonger(na)]
        \question white \f[vit(t,a)]
        \question green \f[grön(t,a)]
        \question gray \f[grå(tt,a)]
        \question orange \f[orange(\_,a)]
        \question striped \f[randig(t,a)]
        \question checkered \f[rutig(t,a)]
        \question yellow \f[gul(t,a)]
        \question brown \f[brun(t,a)]
        \question black \f[svart(a)]
        \question pink \f[rosa]
        \question spotted \f[prickig(t,a)]
        \question  \f[]
        \question  \f[]
        \question  \f[]
        \question  \f[]
        \question  \f[]
        \question  \f[]
        \question  \f[]
        \question  \f[]
        \question  \f[]
        \question  \f[]
        \question  \f[]
        \question  \f[]
        \question  \f[]
        \question  \f[]
        \question  \f[]
        \question  \f[]
        \question  \f[]
        \question  \f[]
        \question  \f[]
        \question  \f[]
        \question  \f[]
        \question  \f[]
        \question  \f[]
        \question  \f[]
        \question  \f[]
        \question  \f[]
        \question  \f[]
        \question  \f[]
        \question  \f[]
        \question  \f[]
        \question  \f[]
        \question  \f[]
        \question  \f[]
        \question  \f[]
        \question  \f[]
        \question  \f[]
        \question  \f[]
        \question  \f[]
        \question  \f[]
        \question  \f[]
        \question  \f[]
        \question  \f[]
        \question  \f[]
        \question  \f[]
        \question  \f[]
        \question  \f[]
        \question  \f[]
        \question  \f[]
        \question  \f[]
        \question  \f[]
        \question  \f[]
        \question  \f[]
        \question  \f[]
        \question  \f[]
        \question  \f[]
        \question  \f[]
        \question  \f[]
        \question  \f[]
        \question  \f[]
        \question  \f[]
        \question  \f[]
        \question  \f[]
        \question  \f[]
        \question  \f[]
        \question  \f[]
        \question  \f[]
        \question  \f[]
        \question  \f[]
        \question  \f[]
        \question  \f[]
        \question  \f[]
        \question  \f[]
        \question  \f[]
        \question  \f[]
        \question  \f[]
        \question  \f[]
        \question  \f[]
        \question  \f[]
        \question  \f[]
        \question  \f[]
        \question  \f[]
        \question  \f[]
        \question  \f[]
        \question  \f[]
        \question  \f[]
        \question  \f[]
        \question  \f[]
        \question  \f[]
        \question  \f[]
        \question  \f[]
        \question  \f[]
        \question  \f[]
        \question  \f[]
        \question  \f[]
        \question  \f[]
        \question  \f[]
        \question  \f[]
        \question  \f[]
        \question  \f[]
        \question  \f[]
        \question  \f[]
        \question  \f[]
        \question  \f[]
        \question  \f[]
        \question  \f[]
        \question  \f[]
        \question  \f[]
        \question  \f[]
        \question  \f[]
        \question  \f[]
        \question  \f[]
        \question  \f[]
    \end{multicols}
\end{questions}

    \uplevel{\bfseries Fill in the correct Swedish word (RivStart1, Kapitel 8)}

\begin{flushleft}
    \textbf{Phrases and sentences}
    \begin{itemize}
        \item Titta på utsikten. (Look at the view.)
        \item Gå på museum. (Go to a museum.)
    \end{itemize}
\end{flushleft}

\begin{center}
    \begin{adjustwidth}{-2cm}{}
        \begin{tabular}{|c c c c c c|}
            \hline
            friluftsmuseum(et) & typisk(t,a) & nordisk(t,a) & vilda djur & björn(en,ar,arna) & säl(en,ar,arna) \\
            lodjur(et,\_,en) & varg(en,ar,arna) & skärgård(en,ar,arna) & norr & öster & söder \\
            väster & ö(n,ar,arna) & båt(en,ar,arna) & turist(en,er) & slott(et,\_,en) & historia(en,er,erna) \\
            slottsteater(n) & första gången & planera(r/de/t) & väder(vädret) & packa(r/de/t) & prognos(en,er,erna) \\
            regn(et,\_,en) & regnkläder(na) & solglasögon(en) & varm tröja & grej(en,er,erna) & myggmedel(let,\_,len) \\
            hemma hos & forum(et,\_,en) & mitt i stan & åka(er) runt & att göra-lista & promenera i\ldots \\
            designprylar & Moderna museet & paddla(r) kanot &  &  &  \\
             &  &  &  &  &  \\
             &  &  &  &  &  \\
             &  &  &  &  &  \\
             &  &  &  &  &  \\
             &  &  &  &  &  \\
             &  &  &  &  &  \\
             &  &  &  &  &  \\
             &  &  &  &  &  \\
             &  &  &  &  &  \\
             &  &  &  &  &  \\
             &  &  &  &  &  \\
             &  &  &  &  &  \\
             &  &  &  &  &  \\
             &  &  &  &  &  \\
             &  &  &  &  &  \\
             &  &  &  &  &  \\
             &  &  &  &  &  \\
             &  &  &  &  &  \\
             &  &  &  &  &  \\
             &  &  &  &  &  \\
             &  &  &  &  &  \\
             &  &  &  &  &  \\
             &  &  &  &  &  \\
             &  &  &  &  &  \\
             &  &  &  &  &  \\
             &  &  &  &  &  \\
             &  &  &  &  &  \\
             &  &  &  &  &  \\
             &  &  &  &  &  \\
             &  &  &  &  &  \\
             &  &  &  &  &  \\
             &  &  &  &  &  \\
             &  &  &  &  &  \\
             &  &  &  &  &  \\
             &  &  &  &  &  \\
             &  &  &  &  &  \\
             &  &  &  &  &  \\
             &  &  &  &  &  \\
             &  &  &  &  &  \\
             &  &  &  &  &  \\
             &  &  &  &  &  \\
             &  &  &  &  &  \\
             &  &  &  &  &  \\
            \hline
        \end{tabular}
    \end{adjustwidth}
\end{center}

\begin{questions}
    \begin{multicols}{2}
        \raggedcolumns
        \question open-air museum \f[friluftsmuseum(et)]
        \question typical \f[typisk(t,a)]
        \question Nordic \f[nordisk(t,a)]
        \question wild animal \f[vilda djur]
        \question bear \f[björn(en,ar,arna)]
        \question seal \f[säl(en,ar,arna)]
        \question lynx \f[lodjur(et,\_,en)]
        \question wolf \f[varg(en,ar,arna)]
        \question archipelago \f[skärgård(en,ar,arna)]
        \question north \f[norr]
        \question east \f[öster]
        \question south \f[söder]
        \question west \f[väster]
        \question island \f[ö(n,ar,arna)]
        \question boat \f[båt(en,ar,arna)]
        \question tourist \f[turist(en,er,erna)]
        \question palace \f[slott(et,\_,en)]
        \question history \f[historia(en,er,erna)]
        \question palace theater \f[slottsteater(n)]
        \question first time \f[första gången]
        \question to plan \f[planera(r/de/t)]
        \question weather \f[väder(vädret)]
        \question to pack \f[packa(r/de/t)]
        \question forecast \f[prognos(en,er,erna)]
        \question rain \f[regn(et,\_,en)]
        \question rainclothes \f[regnkläder(na)]
        \question sunglasses \f[solglasögon(en)]
        \question warm sweater \f[varm tröja]
        \question thing \f[grej(en,er,erna)]
        \question mosquito repellent \f[myggmedel(let)]
        \question at the home of \f[hemma hos]
        \question forum \f[forum(et,\_,en)]
        \question in the middle of the city \f[mitt i stan]
        \question travel around \f[åka(er) runt]
        \question to-do list \f[att göra-lista]
        \question take a walk in \ldots \f[promenera i]
        \question design gadgets \f[designprylar]
        \question the museum of Modern Art \f[Moderna museet]
        \question go canoeing \f[paddla(r) kanot]
        \question  \f[]
        \question  \f[]
        \question  \f[]
        \question  \f[]
        \question  \f[]
        \question  \f[]
        \question  \f[]
        \question  \f[]
        \question  \f[]
        \question  \f[]
        \question  \f[]
        \question  \f[]
        \question  \f[]
        \question  \f[]
        \question  \f[]
        \question  \f[]
        \question  \f[]
        \question  \f[]
        \question  \f[]
        \question  \f[]
        \question  \f[]
        \question  \f[]
        \question  \f[]
        \question  \f[]
        \question  \f[]
        \question  \f[]
        \question  \f[]
        \question  \f[]
        \question  \f[]
        \question  \f[]
        \question  \f[]
        \question  \f[]
        \question  \f[]
        \question  \f[]
        \question  \f[]
        \question  \f[]
        \question  \f[]
        \question  \f[]
        \question  \f[]
        \question  \f[]
        \question  \f[]
        \question  \f[]
        \question  \f[]
        \question  \f[]
        \question  \f[]
        \question  \f[]
        \question  \f[]
        \question  \f[]
        \question  \f[]
        \question  \f[]
        \question  \f[]
        \question  \f[]
        \question  \f[]
        \question  \f[]
        \question  \f[]
        \question  \f[]
        \question  \f[]
        \question  \f[]
        \question  \f[]
        \question  \f[]
        \question  \f[]
        \question  \f[]
        \question  \f[]
        \question  \f[]
        \question  \f[]
        \question  \f[]
        \question  \f[]
        \question  \f[]
        \question  \f[]
        \question  \f[]
        \question  \f[]
        \question  \f[]
        \question  \f[]
        \question  \f[]
        \question  \f[]
        \question  \f[]
        \question  \f[]
        \question  \f[]
        \question  \f[]
        \question  \f[]
        \question  \f[]
        \question  \f[]
        \question  \f[]
        \question  \f[]
        \question  \f[]
        \question  \f[]
        \question  \f[]
        \question  \f[]
        \question  \f[]
        \question  \f[]
        \question  \f[]
        \question  \f[]
        \question  \f[]
        \question  \f[]
        \question  \f[]
        \question  \f[]
        \question  \f[]
        \question  \f[]
        \question  \f[]
        \question  \f[]
        \question  \f[]
        \question  \f[]
        \question  \f[]
        \question  \f[]
        \question  \f[]
        \question  \f[]
        \question  \f[]
        \question  \f[]
        \question  \f[]
        \question  \f[]
        \question  \f[]
        \question  \f[]
        \question  \f[]
        \question  \f[]
        \question  \f[]
        \question  \f[]
        \question  \f[]
        \question  \f[]
        \question  \f[]
        \question  \f[]
        \question  \f[]
        \question  \f[]
        \question  \f[]
        \question  \f[]
        \question  \f[]
        \question  \f[]
        \question  \f[]
        \question  \f[]
        \question  \f[]
        \question  \f[]
        \question  \f[]
        \question  \f[]
        \question  \f[]
        \question  \f[]
        \question  \f[]
        \question  \f[]
        \question  \f[]
        \question  \f[]
        \question  \f[]
        \question  \f[]
        \question  \f[]
        \question  \f[]
        \question  \f[]
        \question  \f[]
        \question  \f[]
        \question  \f[]
        \question  \f[]
        \question  \f[]
        \question  \f[]
        \question  \f[]
        \question  \f[]
        \question  \f[]
        \question  \f[]
        \question  \f[]
        \question  \f[]
        \question  \f[]
        \question  \f[]
        \question  \f[]
        \question  \f[]
        \question  \f[]
        \question  \f[]
        \question  \f[]
        \question  \f[]
        \question  \f[]
        \question  \f[]
        \question  \f[]
        \question  \f[]
        \question  \f[]
        \question  \f[]
        \question  \f[]
        \question  \f[]
        \question  \f[]
        \question  \f[]
        \question  \f[]
        \question  \f[]
        \question  \f[]
        \question  \f[]
        \question  \f[]
        \question  \f[]
        \question  \f[]
        \question  \f[]
        \question  \f[]
        \question  \f[]
        \question  \f[]
        \question  \f[]
        \question  \f[]
        \question  \f[]
        \question  \f[]
        \question  \f[]
        \question  \f[]
        \question  \f[]
        \question  \f[]
        \question  \f[]
        \question  \f[]
        \question  \f[]
        \question  \f[]
        \question  \f[]
        \question  \f[]
        \question  \f[]
        \question  \f[]
        \question  \f[]
        \question  \f[]
        \question  \f[]
        \question  \f[]
        \question  \f[]
        \question  \f[]
        \question  \f[]
        \question  \f[]
        \question  \f[]
        \question  \f[]
        \question  \f[]
        \question  \f[]
        \question  \f[]
        \question  \f[]
        \question  \f[]
        \question  \f[]
        \question  \f[]
        \question  \f[]
        \question  \f[]
        \question  \f[]
        \question  \f[]
        \question  \f[]
        \question  \f[]
        \question  \f[]
        \question  \f[]
        \question  \f[]
        \question  \f[]
        \question  \f[]
        \question  \f[]
        \question  \f[]
        \question  \f[]
    \end{multicols}
\end{questions}

    \uplevel{\bfseries Fill in the correct Swedish word (RivStart1, Kapitel 9)}

\begin{flushleft}
    \textbf{Phrases and sentences}
    \begin{itemize}
        \item Första till vänster. (First on the left)
        \item Andra till höger. (Second on the right)
        \item Hur lång tid tar det? (How long does it take?)
        \item Kommer du ihåg? (Do you remember?)
    \end{itemize}
\end{flushleft}

\begin{center}
    \begin{adjustwidth}{-2.5cm}{}
        \begin{tabular}{|c c c c c c|}
            \hline
            transport(en,er,erna) & spår(et,\_,en) & vagn(en,ar,arna) & max(maximalt) & hållplats(en,er,erna) & snabb(t,a) \\
            mack(en,ar,arna) & rondell(en,er,erna) & checka(r/de/t) & handbagage(et) & avgångstid(en,er,erna) & stressa(r/de/t) \\
            färja(n,or,orna) & gå(gick/gått) av & nästa station & linje(n,er,erna) & det beror på & trafik(en) \\
            fast pris & taxameter(n) & etta(n,or,orna) & tvåa & trea & fyra \\
            femma(n) & sexa(n) & sjua(n,or) & nia(n) & tia(n) & elva(n)  \\
            tolva(n) & gå i tvåan & nollåtta & hiss(en,ar,arna) & ingenting & jättenöjd(t/a) \\
            sjuttiosexa & åttiofemma & femtiosjua & ung(yngre/st) & växla(r) en tjuga & tia(mynt) \\
            femma(mynt) & en sexa whiskey & Vilken tur! & skostorlek(en) & spårvagn(en,ar,arna) & födelseår(et) \\
            gatunummer(ret) & våningsplan(et,\_,en) & centiliter alkohol & drink(en,ar) & tärningsslag(et,\_,en) & årskurs(en) \\
            flyg(et,\_,en) & åka(er) tillbaka & hälsa(r/de/t) på & partikelverb(et) & paket(et,\_,en) & obetonad(t/a) \\
            pendeltåg(et,\_,en) & ta flyget & camping(en) & Öresundsbron & mysig(t/a) & framtidsadverb \\
            föreläsning(en,ar) &  &  &  &  &  \\
             & & & & &  \\
             & & & & &  \\
             & & & & &  \\
             & & & & &  \\
             & & & & &  \\
             & & & & &  \\
             & & & & &  \\
             & & & & &  \\
             & & & & &  \\
             & & & & &  \\
             & & & & &  \\
             & & & & &  \\
             & & & & &  \\
             & & & & &  \\
             & & & & &  \\
             & & & & &  \\
             & & & & &  \\
             & & & & &  \\
             & & & & &  \\
             & & & & &  \\
             & & & & &  \\
             & & & & &  \\
             & & & & &  \\
             & & & & &  \\
             & & & & &  \\
             & & & & &  \\
             & & & & &  \\
             & & & & &  \\
             & & & & &  \\
             & & & & &  \\
             & & & & &  \\
             & & & & &  \\
             & & & & &  \\
             & & & & &  \\
             & & & & &  \\
             & & & & &  \\
             & & & & &  \\
             & & & & &  \\
             & & & & &  \\
            \hline
        \end{tabular}
    \end{adjustwidth}
\end{center}

\begin{questions}
    \begin{multicols}{2}
        \raggedcolumns
        \question transport \f[transport(en,er,erna)]
        \question rail \f[spår(et,\_,en)]
        \question carriage \f[vagn(en,ar,arna)]
        \question max \f[maximalt]
        \question bus station \f[hållplats(en)]
        \question quick \f[snabb(t/a)]
        \question gas station \f[mack(en,ar,arna)]
        \question roundabout \f[rondell(en,er,erna)]
        \question to check \f[checka(r/de/t)]
        \question hand luggage \f[handbagage(et)]
        \question departure time \f[avgångstid(en)]
        \question to stress \f[stressa(r/de/t)]
        \question ferry \f[färja(n,or,orna)]
        \question get off \f[gå av]
        \question next station \f[nästa station]
        \question line \f[linje(n,er,erna)]
        \question it depends \f[det beror på]
        \question traffic \f[trafik(en)]
        \question fixed price \f[fast pris]
        \question taximeter \f[taxameter(n)]
        \question one \f[etta(n)]
        \question two \f[tvåa(n)]
        \question three \f[trea]
        \question four \f[fyra]
        \question five \f[femma(n)]
        \question six \f[sexa(n)]
        \question seven \f[sjua(n)]
        \question eight \f[åtta(n)]
        \question nine \f[nia(n)]
        \question ten \f[tia(n)]
        \question eleven \f[elva(n)]
        \question twelfve \f[tolva(n)]
        \question in second grade \f[gå in tvåan]
        \question Stockholmer($08$) \f[nollåtta]
        \question elevator \f[hiss(en,ar,arna)]
        \question nothing \f[ingenting]
        \question very satisfied \f[jättenöjd(t/a)]
        \question ppl born in $76$ \f[sjuttiosexa(n)]
        \question ppl born in $85$ \f[åttiofemma(n)]
        \question ppl born in $57$ \f[femtiosjua(n)]
        \question young \f[ung(yngre/yngst)]
        \question break change for a twenty \f[växla(r) en tjuga]
        \question ten-kronor coin \f[tia(mynt)]
        \question five-kronor coin \f[femma(mynt)]
        \question $6$ cl of whiskey \f[en sexa whiskey]
        \question What luck! \f[Vilken tur!]
        \question centiliters of alcohol \f[centiliter alkohol]
        \question drink \f[drink(en)]
        \question dice roll \f[tärningsslag(et,\_,en)]
        \question year, grade \f[årskurs(en,er)]
        \question flight \f[flyg(et,\_,en)]
        \question go back \f[åka(er) tillbaka]
        \question to visit \f[hälsa(r/de/t) på]
        \question phrasal verb \f[partikelverb(et)]
        \question package \f[paket(et,\_,en)]
        \question unstressed \f[obetonad(t/a)]
        \question commute train \f[pendeltåg(et,\_,en)]
        \question fly,take a plane \f[ta flyget]
        \question camping \f[camping(en)]
        \question Öresund bridge \f[Öresundsbron]
        \question cozy \f[mysig(t,a)]
        \question adverbs for future \f[framtidsadverb(et)]
        \question lecture \f[föreläsning(en)]
        \question  \f[]
        \question  \f[]
        \question  \f[]
        \question  \f[]
        \question  \f[]
        \question  \f[]
        \question  \f[]
        \question  \f[]
        \question  \f[]
        \question  \f[]
        \question  \f[]
        \question  \f[]
        \question  \f[]
        \question  \f[]
        \question  \f[]
        \question  \f[]
        \question  \f[]
        \question  \f[]
        \question  \f[]
        \question  \f[]
        \question  \f[]
        \question  \f[]
        \question  \f[]
        \question  \f[]
        \question  \f[]
        \question  \f[]
        \question  \f[]
        \question  \f[]
        \question  \f[]
        \question  \f[]
        \question  \f[]
        \question  \f[]
        \question  \f[]
        \question  \f[]
        \question  \f[]
        \question  \f[]
        \question  \f[]
        \question  \f[]
        \question  \f[]
        \question  \f[]
        \question  \f[]
        \question  \f[]
        \question  \f[]
        \question  \f[]
        \question  \f[]
        \question  \f[]
        \question  \f[]
        \question  \f[]
        \question  \f[]
        \question  \f[]
        \question  \f[]
        \question  \f[]
        \question  \f[]
        \question  \f[]
        \question  \f[]
        \question  \f[]
        \question  \f[]
        \question  \f[]
        \question  \f[]
        \question  \f[]
        \question  \f[]
        \question  \f[]
        \question  \f[]
        \question  \f[]
        \question  \f[]
        \question  \f[]
        \question  \f[]
        \question  \f[]
        \question  \f[]
        \question  \f[]
        \question  \f[]
        \question  \f[]
        \question  \f[]
        \question  \f[]
        \question  \f[]
        \question  \f[]
        \question  \f[]
        \question  \f[]
        \question  \f[]
        \question  \f[]
        \question  \f[]
        \question  \f[]
        \question  \f[]
        \question  \f[]
        \question  \f[]
        \question  \f[]
        \question  \f[]
        \question  \f[]
        \question  \f[]
        \question  \f[]
        \question  \f[]
        \question  \f[]
        \question  \f[]
        \question  \f[]
        \question  \f[]
        \question  \f[]
        \question  \f[]
        \question  \f[]
        \question  \f[]
        \question  \f[]
        \question  \f[]
        \question  \f[]
        \question  \f[]
        \question  \f[]
        \question  \f[]
        \question  \f[]
        \question  \f[]
        \question  \f[]
        \question  \f[]
        \question  \f[]
        \question  \f[]
        \question  \f[]
        \question  \f[]
        \question  \f[]
        \question  \f[]
        \question  \f[]
        \question  \f[]
        \question  \f[]
        \question  \f[]
        \question  \f[]
        \question  \f[]
        \question  \f[]
        \question  \f[]
        \question  \f[]
        \question  \f[]
        \question  \f[]
        \question  \f[]
        \question  \f[]
        \question  \f[]
        \question  \f[]
        \question  \f[]
        \question  \f[]
        \question  \f[]
        \question  \f[]
        \question  \f[]
        \question  \f[]
        \question  \f[]
        \question  \f[]
        \question  \f[]
        \question  \f[]
        \question  \f[]
        \question  \f[]
        \question  \f[]
        \question  \f[]
        \question  \f[]
        \question  \f[]
        \question  \f[]
        \question  \f[]
        \question  \f[]
    \end{multicols}
\end{questions}

    \uplevel{\bfseries Fill in the correct Swedish word (RivStart1, Kapitel 10)}

\begin{flushleft}
    \textbf{Phrases and sentences}
    \begin{itemize}
        \item 
    \end{itemize}
\end{flushleft}

\begin{center}
    \begin{adjustwidth}{-2.7cm}{}
        \begin{tabular}{|c c c c c c|}
            \hline
            fakta om & skandinavisk(t,a) & halvö(n,ar) & polcirkel(n) & genom & platt(a) \\
            kustlinje(n,er) & sjö(n,ar,arna) & kust(en,er,erna) & tiotusental & Gotland & Öland \\
            Östersjön & Europa & ytan & tredjedel(en,ar,arna) & ursprung(et,\_,en) & befolkning(en,ar) \\
            same(n,r,rna) & samisk(t/a) & parlament(et,\_,en) & istid(en,er,erna) & meänkieli & fattig(t,a) \\
            därför & emigrera(r/de/t) & industri(n,er,erna) & arbetskraft(en) & jordbruk(et,\_,ene) & Sydeuropa \\
            ändra(r/de/t) & politik(en) & flykting(en,ar,arna) & samtidig(t/a) & invandring(en) & asyl \\
            Latinamerika & Mellanöstern & EU-medborgare(n) & femtedel(en) & export(en,er) & trä(et,t) \\
            pappersmassa(n) & järn(et,\_,en) & stål(et,\_,en) & elektronik(en) & telekom & artist(en,er) \\
            vara(n,or,orna) & skogsvaror & mineralvaror & kemivaror & energivaror & övriga \\
            verkstadsprodukter & ställa(er/de/t) & invånare(n) & de flesta & rik(t/a) & naturresurs(en,er) \\
            exportera(r/de/t) & säker(t,säkra) & exakt(a) & komma ihåg & kom/kommit & fel(et,\_,en) \\
            politiker(n,\_,na) & diplomat(en,er) & vetenskapsman(nen) & vetenskapsmän(nen) & popstjärna(n,or) & regissör(en,er) \\
            helgon(et,\_,en) & idrottare(n) & operasångare(n) & operasångerska(n) & millennium(et,er) & exemplar(et,\_,en) \\
            det mesta & normal(t,a) & utlänning(en,ar) & studenttidning(en) & intervju(n,er) & mil(en) \\
            kilometer(n) & fläskfile(n,er) & jättekonstig(t,a) & mitt på dagen & barnvagn(en,ar) & kaffe latte \\
            jättegullig(t,a) & föräldraledig(t,a) & i stället för & Österrike & praktisk(t,a) & köttbulle(n,ar) \\
            sylt(en) & pasta(n) & Frankrike & tobak(en) & läpp(en,ar,arna) & Usch! \\
            luft(en) & astma(n) & Lettland & gullig(t,a) & hemsk(t,a) & jiddish \\
            kvadratkilometer(n) & konstitutionell & monarki & Kebnekaise & romani chib & huvudstad(en) \\
            officiell(t/a) & statsskick(et) & minoritetsspråk(et) & valuta(n,or) & regent(en,er) & täthet \\
            flod(et,\_,en) & berg(et,\_,en) & landsnummer(ret) & brutto & nationalprodukt &  \\
             &  &  &  &  &  \\
             &  &  &  &  &  \\
             &  &  &  &  &  \\
             &  &  &  &  &  \\
             &  &  &  &  &  \\
             &  &  &  &  &  \\
             &  &  &  &  &  \\
             &  &  &  &  &  \\
             &  &  &  &  &  \\
             &  &  &  &  &  \\
             &  &  &  &  &  \\
             &  &  &  &  &  \\
             &  &  &  &  &  \\
             &  &  &  &  &  \\
             &  &  &  &  &  \\
             &  &  &  &  &  \\
             &  &  &  &  &  \\
             &  &  &  &  &  \\
             &  &  &  &  &  \\
             &  &  &  &  &  \\
             &  &  &  &  &  \\
            \hline
        \end{tabular}
    \end{adjustwidth}
\end{center}

\begin{questions}
    \begin{multicols}{2}
        \raggedcolumns
        \question facts about \f[fakta om]
        \question Scandinavia \f[skandinavisk(t/a)]
        \question peninsula \f[halvö(n,ar)]
        \question through \f[genom]
        \question Arctic circle \f[polcirkel(n)]
        \question flat \f[platt(a)]
        \question coast line \f[kustlinje(n,er)]
        \question lake \f[sjö(n,ar,arna)]
        \question coast \f[kust(en,er,erna)]
        \question tens of thousands \f[tiotusental]
        \question Gotland, island in the Baltic \f[Gotland]
        \question Öland, island in the Baltic \f[Öland]
        \question Baltic Sea \f[Östersjön]
        \question Europe \f[Europa]
        \question the surface \f[ytan]
        \question a third \f[tredjedel(n,ar,arna)]
        \question origin \f[unsprung(et,\_,en)]
        \question people \f[befolkning(en,ar,arna)]
        \question Sami people \f[same(n,r,rna)]
        \question Sami \f[samisk(t,a)]
        \question Parliament \f[parlament(et)]
        \question ice age \f[istid(en,er,erna)]
        \question Finnish dialect \f[meänkieli]
        \question poor \f[fattig(t,a)]
        \question therefore \f[därför]
        \question emigrate \f[emigrera(r/de/t)]
        \question industry \f[industri(n,er,erna)]
        \question manpower \f[arbetskraft(en)]
        \question agriculture \f[jordbruk(et,\_,ene)]
        \question Southern Europe \f[Sydeuropa]
        \question to change \f[ändra(r/de/t)]
        \question politics \f[politik(en)]
        \question refugee \f[flykting(en)]
        \question simultaneous \f[samtidig(t/a)]
        \question immigration \f[invandring(en)]
        \question asylum \f[asyl]
        \question Latin America \f[Latinamerika]
        \question Middle East \f[Mellanöstern]
        \question EU citizen \f[EU-medborgare(n)]
        \question fifth \f[femtedel(en)]
        \question export \f[export(en,er)]
        \question wood \f[trä(et,t)]
        \question wood pulp \f[pappersmassa(n)]
        \question iron \f[järn(et,\_,en)]
        \question steel \f[stål(et,\_,en)]
        \question electronics \f[elektronik(en)]
        \question telecommunications \f[telekom]
        \question artist \f[artist(en,er)]
        \question goods \f[vara(n,or,orna)]
        \question forestry goods \f[skogsvaror]
        \question mineral goods \f[mineralvaror]
        \question chemical goods \f[kemivaror]
        \question energy goods \f[energivaror]
        \question others \f[övriga]
        \question manufactured goods \f[verkstadsprodukter]
        \question ask questions \f[ställa fragor]
        \question population/resident \f[invånare(n)]
        \question most people \f[de flesta]
        \question rich \f[rik(t,a)]
        \question natural resources \f[naturresurs(en)]
        \question export \f[exportera(r/de/t)]
        \question certain \f[säker(t/säkra)]
        \question exact \f[exakt(a)]
        \question remember \f[komma ihåg]
        \question came/come \f[kom/kommit]
        \question wrong \f[fel(et,\_,en)]
        \question politician \f[politiker(n,\_,na)]
        \question diplomat \f[diplomat(en,er)]
        \question scientist \f[vetenskapsman(nen)]
        \question scientists \f[ventenskapsmän(nen)]
        \question popstar \f[popstjärna(n)]
        \question director \f[regissör(en,er)]
        \question saint \f[helgon(et,\_,en)]
        \question athlete \f[idrottare(n)]
        \question opera singer \f[operasångare(n)]
        \question opera singer(female) \f[operasångerska(n)]
        \question millenium \f[millennium(et)]
        \question copy \f[exemplar(et,\_,en)]
        \question most \f[de mesta]
        \question normal \f[normal(t,a)]
        \question foreigner \f[utlänning(en,ar)]
        \question student paper \f[studenttidning(en)]
        \question interview \f[interjvu(n,er)]
        \question Swedish mile(10km) \f[mil(en)]
        \question kilometer \f[kilometer(n)]
        \question pork tenderloin \f[fläskfile(n,er)]
        \question really strange \f[jättekonstig(t,a)]
        \question middle of the day \f[mitt på dagen]
        \question baby carriage \f[barnvagn(en,ar)]
        \question latte \f[kaffe latte]
        \question really cute \f[jättegullig(t,a)]
        \question parental leave \f[föräldraledig(t,a)]
        \question instead of \f[i stället för]
        \question Austria \f[Österrike]
        \question practical \f[praktisk(t,a)]
        \question meatball \f[köttbulle(n,ar)]
        \question jam/preserves \f[sylt(en)]
        \question pasta \f[pasta(n)]
        \question France \f[Frankrike]
        \question tobacco \f[tobak(en)]
        \question lip \f[läpp(en,ar,arna)]
        \question Yuck! \f[Usch!]
        \question air \f[luft(en)]
        \question asthma \f[astma(n)]
        \question Latvia \f[Lettland]
        \question cute \f[gullig(t,a)]
        \question horrible \f[hemsk(t,a)]
        \question Yiddish \f[jiddish]
        \question squarekilometer \f[kvadratkilometer(n)]
        \question constitutional \f[konstitutionell]
        \question monarchy \f[monarki]
        \question Kebnekaise \f[Kebnekaise]
        \question Romani language \f[romani chibb]
        \question capital city \f[huvudstad(en)]
        \question official \f[officiell(t,a)]
        \question government \f[statsskick(et)]
        \question minority language \f[minoritetsspråk(et)]
        \question regent \f[regent(en,er)]
        \question currency \f[valuta(n,or)]
        \question density \f[täthet]
        \question river \f[flod(et,\_,en)]
        \question fjäll \f[berg(et,\_,en)]
        \question country code \f[landsnummer(ret)]
        \question gross \f[brutto]
        \question national product \f[nationalprodukt(BNP)]
        \question  \f[]
        \question  \f[]
        \question  \f[]
        \question  \f[]
        \question  \f[]
        \question  \f[]
        \question  \f[]
        \question  \f[]
        \question  \f[]
        \question  \f[]
        \question  \f[]
        \question  \f[]
        \question  \f[]
        \question  \f[]
        \question  \f[]
        \question  \f[]
        \question  \f[]
        \question  \f[]
        \question  \f[]
        \question  \f[]
        \question  \f[]
        \question  \f[]
        \question  \f[]
        \question  \f[]
        \question  \f[]
        \question  \f[]
        \question  \f[]
        \question  \f[]
        \question  \f[]
    \end{multicols}
\end{questions}


    \clearpage
    \bibliographystyle{plain}
    \bibliography{swe_texts}

\end{document}