\documentclass[addpoints,a3paper,11pt]{exam}
\usepackage{natbib}
\usepackage{amsmath}
\usepackage[swedish]{babel}
\usepackage[affil-it]{authblk}
\usepackage[utf8]{inputenc}
\usepackage{multicol,microtype}
\newcommand{\f}[1][{}]{\fillin[#1][0.5in]}
\newcommand{\note}[1]{\textit{#1}}
\pagestyle{headandfoot}
\chead[Leadoo Marketing Technologies \\
    Swedish self-test vocabulary \\
    Son To]{}
\cfoot{Page \thepage\ of \numpages}
\rfoot{\iflastpage{THE END}{Please go on to the next page\ldots}}
\begin{document}
    \title{\bfseries Swedish Master}
    \author{Son To}
    \affil{Leadoo Marketing Technologies
    \thanks{I thank my employer!}}
    \date{25. heinäkuuta 2021}
    \maketitle

    \begin{coverpages}
        This is the review of all vocabularies that appear
        in various texts,
        namely \cite{EngToSwe1},
        in my attempt to master Swedish, once and for all.
    \end{coverpages}

    \begin{center}
        \fbox{\fbox{\parbox{5.5in}{\centering
            This is the self-review made by me using
            the language \TeX\ written by Donald Knuth with
            \LaTeX\ as its successor in implementation. The
            documentation for this review comes from exam.cls
            made by Philip Hirschhorn.}}}
    \end{center}

    \vspace{0.1in}

    \makebox[\textwidth]{Name:\enspace\hrulefill}

    \vspace{0.2in}

    \makebox[\textwidth]{Year of birth:\enspace\hrulefill}

    \uplevel{\bfseries Fill in the correct Swedish word (EngToSwe1, pronounciation)}
\begin{center}
    \begin{tabular}{|c c c c c c|}
        \hline
        en arm & en hand & en katt & ett glas & ett par & en tomat \\
        en boll & en doktor & en klocka & ett kilo & en orm & en ost \\
        \hline
    \end{tabular}
\end{center}

\begin{questions}
    \begin{multicols}{3}
        \raggedcolumns
        \question an arm \fillin
        \question a hand \fillin
        \question a cat \fillin
        \question a ball \fillin
        \question a doctor \fillin
        \question a clock \fillin
        \question a kilogram \fillin
        \question a snake \fillin
        \question a cheese \fillin
    \end{multicols}
\end{questions}
    \uplevel{\bfseries Fill in the correct Swedish word (EngToSwe1, Lektion 1)}
\begin{center}
    \begin{tabular}{|c c c c c c|}
        \hline
        jag & du & han & hon & den & det \\
        vi & ni & de & en lektion & det här & är \\
        en familj & en pappa & i & heter & en mamma & och \\
        lektionen  & familjen & pappan & mamman & en flicka & till \\
        flickan & till & en pojke & pojken & som & en syster \\
        systern & en bror & brodern & har & också & varje \\
        en dag & daggen & går & ut &  med  &  men \\
        inte & vem & själv & vad & en intervju & intervjun \\
        \hline
    \end{tabular}
\end{center}

\begin{questions}
    \begin{multicols}{2}
        \raggedcolumns
        \question I \f[jag]
        \question you(singular) \f[du]
        \question he \f[han]
        \question she \f[hon]
        \question it (en-nouns) \f[den]
        \question it (ett-nouns) \f[det]
        \question we \f[vi]
        \question you(plural) \f[ni]
        \question they \f[de]
        \question a lesson \f[en lektion]
        \question the lession \f[lektionen]
        \question this \f[det här]
        \question am/is/are \f[är]
        \question a family \f[en familj]
        \question a father \f[en pappa]
        \question in \f[in]
        \question am/us/are called \f
        \question a mother \f[en mamma]
        \question and \f[och]
        \question the lession \f[lektionen]
        \question the family \f[familjen]
        \question the father \f[pappan]
        \question the mother \f[mamman]
        \question a girl \f[en flickan]
        \question to, of \f[till]
        \question the girl \f[flickan]
        \question a boy \f[en pojke]
        \question the boy \f[pojken]
        \question who, which, as, like \f[som]
        \question a sister \f[en syster]
        \question the sister \f[systern]
        \question a brother \f[en bror]
        \question the brother \f[brodern]
        \question have/has \f[har]
        \question also \f[också]
        \question every \f[varje]
        \question a day \f[en dag]
        \question the day \f[daggen]
        \question go \f[går]
        \question out \f[ut]
        \question with \f[med]
        \question but \f[men]
        \question not \f[inte]
        \question who \f[vem]
        \question self \f[själv]
        \question what \f[vad]
        \question an interview \f[en intervju]
        \question the interview \f[intervjun]
    \end{multicols}
\end{questions}
    \uplevel{\bfseries Fill in the correct Swedish word (EngToSwe1, Lektion 2)}
\begin{flushleft}
    Also write down infinitive vs present tense verb forms (-a vs -ar/-er).
\end{flushleft}
\begin{center}
    \begin{tabular}{|c c c c c c|}
        \hline
        komma & bo & där & ett hus & från & cykla \\
        ett arbete & en skola & Sverige & vanligt & att & stort \\
        \hline
    \end{tabular}
\end{center}

\begin{questions}
    \begin{multicols}{2}
        \raggedcolumns
        \question to come \f[komma]
        \question to live \f[bo]
    \end{multicols}
\end{questions}
    \uplevel{\bfseries Fill in the correct Swedish word (EngToSwe1, Lektion 3)}
\begin{flushleft}
    Also write down definitive, plural and definitive plural form
    for each word,\\
    if possible, remember:
    \begin{equation*}
        \begin{aligned}
            -a &\Rightarrow -or \Rightarrow -orna & \\
            \ldots &\Rightarrow -ar \Rightarrow -arna & \\
            \text{last syll} &\Rightarrow -er \Rightarrow -erna & \\
            \text{a,i,e,o,u} &\Rightarrow -n \Rightarrow -na & \\
            \ldots &\Rightarrow \underline{\phantom{no}} \Rightarrow -en & \\
        \end{aligned}
        \begin{aligned}
            &\left.\vphantom{\begin{aligned}
                -a &\Rightarrow -or \Rightarrow -orna & \\
                \ldots &\Rightarrow -ar \Rightarrow -arna & \\
                \text{last syll} &\Rightarrow -er \Rightarrow -erna & \\
            \end{aligned}}\right\rbrace\quad\text{en}\\
            &\left.\vphantom{\begin{aligned}
                \text{a,i,e,o,u} &\Rightarrow -n \Rightarrow -na & \\
                \ldots &\Rightarrow \underline{\phantom{no}} \Rightarrow -en & \\
            \end{aligned}}\right\rbrace\quad\text{ett}
        \end{aligned}
    \end{equation*}
\end{flushleft}
\begin{center}
    \begin{tabular}{|c c c c c c|}
        \hline
        grammatik & grammatiken & glosa & sur & glad & mycket \\
        typiskt & barn & alla & ute & säga & förstå \\
        trädgård & här & tycka & roligt & aha & två \\
        minut & något & utbytesstudent & extra & många & fråga \\
        om & akta & dig & lyfta & ben & fot \\
        \hline
    \end{tabular}
\end{center}

\begin{questions}
    \begin{multicols}{2}
        \raggedcolumns
        \question grammar \f[grammatik]
        \question vocabulary word \f[glosa]
        \question sulky, sour \f[sur]
        \question happy, glad \f[glad]
        \question very \f[mycket]
        \question typical \f[typiskt]
        \question child \f[barn]
        \question all \f[alla]
        \question outside \f[ute]
        \question say \f[säga]
        \question understand \f[förstå]
        \question garden \f[trädgård]
        \question here \f[här]
        \question think \f[tycka]
        \question fun \f[roligt]
        \question I see \f[aha]
        \question two \f[två]
        \question a minute \f[minut]
        \question something, anything \f[något]
        \question exchange student \f[utbytesstudent]
        \question extra \f[extra]
        \question many \f[många]
        \question ask \f[fråga]
        \question if, about \f[om]
        \question watch out, mind \f[akta]
        \question yourself, you \f[dig]
        \question to lift \f[lyfta]
        \question leg \f[ben]
        \question foot \f[fot]
    \end{multicols}
\end{questions}

    \clearpage
    \bibliographystyle{plain}
    \bibliography{swe_texts}

\end{document}