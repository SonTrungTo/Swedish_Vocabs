\documentclass[addpoints,a3paper,11pt]{exam}
\usepackage{natbib}
\usepackage{amsmath}
\usepackage[swedish]{babel}
\usepackage[affil-it]{authblk}
\usepackage[utf8]{inputenc}
\usepackage{multicol,microtype}
\newcommand{\f}[1][{}]{\fillin[#1][0.5in]}
\newcommand{\note}[1]{\textit{#1}}
\pagestyle{headandfoot}
\chead[Leadoo Marketing Technologies \\
    Swedish self-test vocabulary \\
    Son To]{}
\cfoot{Page \thepage\ of \numpages}
\rfoot{\iflastpage{THE END}{Please go on to the next page\ldots}}
\begin{document}
    \title{\bfseries Swedish Master}
    \author{Son To}
    \affil{Leadoo Marketing Technologies
    \thanks{I thank my employer!}}
    \date{25. heinäkuuta 2021}
    \maketitle

    \begin{coverpages}
        This is the review of all vocabularies that appear
        in various texts,
        namely \cite{EngToSwe1},
        in my attempt to master Swedish, once and for all.
    \end{coverpages}

    \begin{center}
        \fbox{\fbox{\parbox{5.5in}{\centering
            This is the self-review made by me using
            the language \TeX\ written by Donald Knuth with
            \LaTeX\ as its successor in implementation. The
            documentation for this review comes from exam.cls
            made by Philip Hirschhorn.}}}
    \end{center}

    \vspace{0.1in}

    \makebox[\textwidth]{Name:\enspace\hrulefill}

    \vspace{0.2in}

    \makebox[\textwidth]{Year of birth:\enspace\hrulefill}

    \uplevel{\bfseries Fill in the correct Swedish word (EngToSwe1, pronounciation)}
\begin{center}
    \begin{tabular}{|c c c c c c|}
        \hline
        en arm & en hand & en katt & ett glas & ett par & en tomat \\
        en boll & en doktor & en klocka & ett kilo & en orm & en ost \\
        \hline
    \end{tabular}
\end{center}

\begin{questions}
    \begin{multicols}{3}
        \raggedcolumns
        \question an arm \fillin
        \question a hand \fillin
        \question a cat \fillin
        \question a ball \fillin
        \question a doctor \fillin
        \question a clock \fillin
        \question a kilogram \fillin
        \question a snake \fillin
        \question a cheese \fillin
    \end{multicols}
\end{questions}
    \uplevel{\bfseries Fill in the correct Swedish word (EngToSwe1, Lektion 1)}
\begin{center}
    \begin{tabular}{|c c c c c c|}
        \hline
        jag & du & han & hon & den & det \\
        vi & ni & de & en lektion & det här & är \\
        en familj & en pappa & i & heter & en mamma & och \\
        lektionen  & familjen & pappan & mamman & en flicka & till \\
        flickan & till & en pojke & pojken & som & en syster \\
        systern & en bror & brodern & har & också & varje \\
        en dag & daggen & går & ut &  med  &  men \\
        inte & vem & själv & vad & en intervju & intervjun \\
        \hline
    \end{tabular}
\end{center}

\begin{questions}
    \begin{multicols}{2}
        \raggedcolumns
        \question I \f[jag]
        \question you(singular) \f[du]
        \question he \f[han]
        \question she \f[hon]
        \question it (en-nouns) \f[den]
        \question it (ett-nouns) \f[det]
        \question we \f[vi]
        \question you(plural) \f[ni]
        \question they \f[de]
        \question a lesson \f[en lektion]
        \question the lession \f[lektionen]
        \question this \f[det här]
        \question am/is/are \f[är]
        \question a family \f[en familj]
        \question a father \f[en pappa]
        \question in \f[in]
        \question am/us/are called \f
        \question a mother \f[en mamma]
        \question and \f[och]
        \question the lession \f[lektionen]
        \question the family \f[familjen]
        \question the father \f[pappan]
        \question the mother \f[mamman]
        \question a girl \f[en flickan]
        \question to, of \f[till]
        \question the girl \f[flickan]
        \question a boy \f[en pojke]
        \question the boy \f[pojken]
        \question who, which, as, like \f[som]
        \question a sister \f[en syster]
        \question the sister \f[systern]
        \question a brother \f[en bror]
        \question the brother \f[brodern]
        \question have/has \f[har]
        \question also \f[också]
        \question every \f[varje]
        \question a day \f[en dag]
        \question the day \f[daggen]
        \question go \f[går]
        \question out \f[ut]
        \question with \f[med]
        \question but \f[men]
        \question not \f[inte]
        \question who \f[vem]
        \question self \f[själv]
        \question what \f[vad]
        \question an interview \f[en intervju]
        \question the interview \f[intervjun]
    \end{multicols}
\end{questions}
    \uplevel{\bfseries Fill in the correct Swedish word (EngToSwe1, Lektion 2)}
\begin{flushleft}
    Also write down infinitive vs present tense verb forms (-a vs -ar/-er).
\end{flushleft}
\begin{center}
    \begin{tabular}{|c c c c c c|}
        \hline
        komma & bo & där & ett hus & från & cykla \\
        ett arbete & en skola & Sverige & vanligt & att & stort \\
        \hline
    \end{tabular}
\end{center}

\begin{questions}
    \begin{multicols}{2}
        \raggedcolumns
        \question to come \f[komma]
        \question to live \f[bo]
    \end{multicols}
\end{questions}
    \uplevel{\bfseries Fill in the correct Swedish word (EngToSwe1, Lektion 3)}
\begin{flushleft}
    Also write down definitive, plural and definitive plural form
    for each word,\\
    if possible, remember:
    \begin{equation*}
        \begin{aligned}
            -a &\Rightarrow -or \Rightarrow -orna & \\
            \ldots &\Rightarrow -ar \Rightarrow -arna & \\
            \text{last syll} &\Rightarrow -er \Rightarrow -erna & \\
            \text{a,i,e,o,u} &\Rightarrow -n \Rightarrow -na & \\
            \ldots &\Rightarrow \underline{\phantom{no}} \Rightarrow -en & \\
        \end{aligned}
        \begin{aligned}
            &\left.\vphantom{\begin{aligned}
                -a &\Rightarrow -or \Rightarrow -orna & \\
                \ldots &\Rightarrow -ar \Rightarrow -arna & \\
                \text{last syll} &\Rightarrow -er \Rightarrow -erna & \\
            \end{aligned}}\right\rbrace\quad\text{en}\\
            &\left.\vphantom{\begin{aligned}
                \text{a,i,e,o,u} &\Rightarrow -n \Rightarrow -na & \\
                \ldots &\Rightarrow \underline{\phantom{no}} \Rightarrow -en & \\
            \end{aligned}}\right\rbrace\quad\text{ett}
        \end{aligned}
    \end{equation*}
\end{flushleft}
\begin{center}
    \begin{tabular}{|c c c c c c|}
        \hline
        grammatik & grammatiken & glosa & sur & glad & mycket \\
        typiskt & barn & alla & ute & säga & förstå \\
        trädgård & här & tycka & roligt & aha & två \\
        minut & något & utbytesstudent & extra & många & fråga \\
        om & akta & dig & lyfta & ben & fot \\
        \hline
    \end{tabular}
\end{center}

\begin{questions}
    \begin{multicols}{2}
        \raggedcolumns
        \question grammar \f[grammatik]
        \question vocabulary word \f[glosa]
        \question sulky, sour \f[sur]
        \question happy, glad \f[glad]
        \question very \f[mycket]
        \question typical \f[typiskt]
        \question child \f[barn]
        \question all \f[alla]
        \question outside \f[ute]
        \question say \f[säga]
        \question understand \f[förstå]
        \question garden \f[trädgård]
        \question here \f[här]
        \question think \f[tycka]
        \question fun \f[roligt]
        \question I see \f[aha]
        \question two \f[två]
        \question a minute \f[minut]
        \question something, anything \f[något]
        \question exchange student \f[utbytesstudent]
        \question extra \f[extra]
        \question many \f[många]
        \question ask \f[fråga]
        \question if, about \f[om]
        \question watch out, mind \f[akta]
        \question yourself, you \f[dig]
        \question to lift \f[lyfta]
        \question leg \f[ben]
        \question foot \f[fot]
    \end{multicols}
\end{questions}
    \uplevel{\bfseries Fill in the correct Swedish word (EngToSwe1, Lektion 4)}
\begin{flushleft}
    Apply adjective rules and fill in the correct word form whenever necessary, \\
    \begin{itemize}
        \item stor (no change) $\leftarrow$ \textbf{en(singular)} nouns.
        \item stor\textbf{t} $\leftarrow$ \textbf{ett(singular)} nouns.
        \item stor\textbf{a} $\leftarrow$ \textbf{plural} nouns.
        \item Den stor\textbf{a} katten / Lunströms stora katten $\leftarrow$ \textbf{SPECIFIC} individual/object.
        \item Alexander den stor\textbf{e} $\leftarrow$ \textbf{masculine} pronoun.
    \end{itemize}
\end{flushleft}
\begin{center}
    \begin{tabular}{|c c c c c c|}
        \hline
        vår & en/ett & två & tre & fyra & fem \\
        sex & sju & åtta & nio(nie) & tio(tie) & elva \\
        tolv & tretton & fjorton & femton & sexton & sjutton \\
        arton & nitton & tjugo & packa & resväska & liten \\
        små &&&&& \\
        \hline
    \end{tabular}
\end{center}

\begin{questions}
    \begin{multicols}{3}
        \raggedcolumns
        \question our \f[vår]
        \question one \f[en/ett]
        \question two \f[två]
        \question three \f[tre]
        \question four \f[fyra]
        \question five \f[fem]
        \question six \f[sex]
        \question seven \f[sju]
        \question eight \f[åtta]
        \question nine \f[nio(nie)]
        \question ten \f[tio(tie)]
        \question eleven \f[elva]
        \question twelve \f[tolv]
        \question thirteen \f[tretton]
        \question fourteen \f[fjorton]
        \question fifteen \f[femton]
        \question sixteen \f[sexton]
        \question seventeen \f[sjutton]
        \question eighteen \f[arton]
        \question $19$ \f[nitton]
        \question $20$ \f[tjugo]
        \question little(singular) \f[liten]
        \question little(plural) \f[små]
    \end{multicols}
\end{questions}
    \uplevel{\bfseries Fill in the correct Swedish word (EngToSwe1, Lektion 5)}

    \uplevel{\bfseries Fill in the correct Swedish word (EngToSwe1, Lektion 6)}

\begin{flushleft}
    By looking at the picture in the textbook,
    I write down the appropriate text without actually looking at
    the text itself.
    \begin{itemize}
        \item Han tar på sig sina skor
        \item Hon tar på sig sina skor
        \item Han tar på sig hennes skor
        \item Hon tar på sig hans skor
        \item Han tar på henne sina skor
        \item Hon tar på honom sina skor
        \item Han tar på henne hennes skor
        \item Hon tar på honom hans skor
    \end{itemize}
\end{flushleft}

\begin{center}
    \begin{tabular}{|c c c c c c|}
        \hline
        bil & äntligen & prata & stanna & framför & ur \\
        framme & hjälpa & sedan & in(i) & hall & ta av sig \\
        förvånad & göra/gör & regna & smuts & badrum & visa \\
        stänga/stänger & dörr & vänta & golv & bakom & bära/bär \\
        lägga/lägger & byrå & hänga/hänger & garderob & snygg & äta/äter \\
        bord &&&&& \\
        \hline
    \end{tabular}
\end{center}

\begin{questions}
    \begin{multicols}{2}
        \raggedcolumns
        \question car \f[bil,bilen,bilar,bilarna]
        \question to speak \f[prata,pratar]
        \question finally \f[äntligen]
        \question to stand/stop \f[stanna,stannar]
        \question in front of \f[framför]
        \question out of \f[ur]
        \question at the destination \f[framme]
        \question to help \f[hjälpa,hjälper]
        \question then \f[sedan]
        \question into \f[in(i)]
        \question hall \f[hall,hallen,hallar,hallarna]
        \question undress/take off \f[ta av sig]
        \question surprised \f[förvånad,förvånat,förvånade]
        \question to do/make \f[göra,gör]
        \question to rain \f[regna,regnar]
        \question dirt \f[smuts,smutsen]
        \question bathroom \f[badrum,badrummet,badrum,badrummen]
        \question to show \f[visa,visar]
        \question to close \f[stänga/stänger]
        \question door \f[dörr,dörren,dörrar,dörrarna]
        \question to wait \f[vänta,väntar]
        \question behind \f[bakom]
        \question floor \f[golv,golvet,golv,golven]
        \question to carry \f[bära,bär]
        \question to lay,put \f[lägga,lägger]
        \question chest of drawers \f[byrå,byrån,byråar,byråarna]
        \question to hang \f[hänga,hänger]
        \question closet,wardrobe \f[garderob,garderoben,garderober,garderoberna]
        \question good-looking \f[snygg,snyggt,snygga]
        \question to eat \f[äta,äter]
        \question table \f[bord,bordet,bord,borden]
    \end{multicols}
\end{questions}
    \uplevel{\bfseries Fill in the correct Swedish word (EngToSwe1, Lektion 7)}

\begin{flushleft}
    \textbf{Imperative form of the verb}
    \begin{itemize}
        \item att stanna $\Rightarrow$ jag stannar $\Rightarrow$ stanna!
        \item att hjälpa $\Rightarrow$ jag hjälper $\Rightarrow$ hjälp!
    \end{itemize}
\end{flushleft}

\begin{center}
    \begin{tabular}{|c c c c c c|}
        \hline
        tjugoett & tjugotvå & tjugotre & tjugofyra & tjugofem & tjugosex \\
        \hline
    \end{tabular}
\end{center}

\begin{questions}
    \begin{multicols}{2}
        \raggedcolumns
        \question twenty-one \f[tjugoett]
    \end{multicols}
\end{questions}
    \uplevel{\bfseries Fill in the correct Swedish word (EngToSwe1, Lektion 8)}

\begin{flushleft}
    \textbf{Comparison of \underline{adjectives}} \\
    \textsl{Short adjectives} (--are, --ast ($+$ --e(definite)))
    \begin{itemize}
        \item lätt $\Rightarrow$ lättare $\Rightarrow$ lättast
        \item varm $\Rightarrow$ varmare $\Rightarrow$ varmast
        \item den varmaste dagen
    \end{itemize}

    \textsl{Long and --isk adjectives} (mer/mera eller mest)
    \begin{itemize}
        \item estetisk $\Rightarrow$ mer estetisk $\Rightarrow$ mest estetisk
        \item förvånad $\Rightarrow$ mer förvånad $\Rightarrow$ mest förvånad
    \end{itemize}

    \textsl{Some irregular adjectives} (definite: --a or --e(masculine))
    \begin{itemize}
        \item bra $\Rightarrow$ bättre $\Rightarrow$ bäst
        \item dålig $\Rightarrow$ sämre $\Rightarrow$ sämst
        \item ung $\Rightarrow$ yngre $\Rightarrow$ yngst
        \item gammal $\Rightarrow$ äldre $\Rightarrow$ äldst
        \item liten $\Rightarrow$ mindre $\Rightarrow$ minst
        \item stor $\Rightarrow$ större $\Rightarrow$ störst
    \end{itemize}
    Det största huset. Den yngsta kvinnan. Den yngste mannen.

    mycket $\Rightarrow$ mer $\Rightarrow$ mest \\
    många $\Rightarrow$ fler $\Rightarrow$ flest
\end{flushleft}

\begin{center}
    \begin{tabular}{|c c c c c c|}
        \hline
        måndag &&&&& \\
        \hline
    \end{tabular}
\end{center}

\begin{questions}
    \begin{multicols}{2}
        \raggedcolumns
        \question Monday \f[måndag]
    \end{multicols}
\end{questions}
    \uplevel{\bfseries Fill in the correct Swedish word (EngToSwe1, Lektion 9)}

\begin{flushleft}
    \textbf{Positions}
    \begin{itemize}
        \item går \underline{in} $\Rightarrow$ goes inside from outside
        \item är \underline{inne} $\Rightarrow$ is inside
        \item går \underline{\underline{ut} ur} $\Rightarrow$ go outside from inside
        \item är \underline{\underline{ute} ur} $\Rightarrow$ is away from
    \end{itemize}

    \textbf{Predicate}
    \begin{itemize}
        \item Vi
        $\underbrace{
            \overbrace{\text{måste}}^{\text{finite}} 
            \overbrace{\text{gå}}^{\text{infinite}}
        }_{\text{predicate}}$
        till skolan nu.
    \end{itemize}

    \textbf{Particle verbs}
    \begin{itemize}
        \item tala (om) = talk (about) / tala \textbf{om} = tell
        \item tycka (om) = think (about) / tycka \textbf{om} = like
        \item titta (på) = to look (at) / titta \textbf{på} = watch
        \item hälsa (på) = to say hello (to) / hälsa \textbf{på} = visit
    \end{itemize}
\end{flushleft}

\begin{center}
    \begin{tabular}{|c c c c c c|}
        \hline
        skog & inne & tala/talar &  &  &  \\
        \hline
    \end{tabular}
\end{center}

\begin{questions}
    \begin{multicols}{2}
        \raggedcolumns
        \question forest \f[skog,skogen,skogar,skogarna]
        \question inside \f[inne]
        \question talk, tell \f[tala/talar]
    \end{multicols}
\end{questions}
    \uplevel{\bfseries Fill in the correct Swedish word (EngToSwe1, Lektion 10)}

\begin{flushleft}
    \textbf{Vad är klockan?}
    \begin{itemize}
        \item Klockan är tre.
        \item Klockan är fem över tre.
        \item Klockan är tio över tre.
        \item Klockan är kvart över tre.
        \item Klockan är tjugo över tre.
        \item Klockan är fem i halv fyra.
        \item Klockan är halv fyra.
        \item Klockan är fem över halv fyra.
    \end{itemize}
\end{flushleft}
    \uplevel{\bfseries Fill in the correct Swedish word (EngToSwe1, Lektion 11)}

\begin{flushleft}
    \textbf{Reverse when:}
    \begin{itemize}
        \item $\underbrace{\text{Nu}}_{\text{starting word}}$ förstår jag.
        \item $\underbrace{\text{När du förklarar }}_{\text{starting sub-clause}}$ förstår jag
        \item Förstår du vad jag säger?
    \end{itemize}
\end{flushleft}

\begin{center}
    \begin{tabular}{|c c c c c c|}
        \hline
        ett huvud &  &  &  &  &  \\
        \hline
    \end{tabular}
\end{center}

\begin{questions}
    \begin{multicols}{2}
        \raggedcolumns
        \question head \f[huvud,huvudet,huvuden,huvudena]
    \end{multicols}
\end{questions}

    \uplevel{\bfseries Fill in the correct Swedish word (EngToSwe1, Lektion 12)}

\begin{flushleft}
    \textbf{Location of \textsl{inte/aldrig}}
    \begin{itemize}
        \item V $+$ inte; Missan går inte ut nu. Missan vill inte gå ut nu.
        \item Reverse: S $+$ inte; Nu går Missan inte ut. Nu vill Missan inte gå ut.
        \item Sub-clauses: inte $+$ V; Jag vet att Missan inte går ut nu. Jag vet att Missan inte vill gå ut nu.
        \item väl, nog, kanske\ldots $+$ inte; Missan går nog inte ut nu. Missan vill nog inte gå ut nu.
    \end{itemize}
\end{flushleft}

\begin{center}
    \begin{tabular}{|c c c c c c|}
        \hline
        en gaffel & en tallrik & en kniv & en sked & en servett & djup/t/a \\
        djup tallrik & ett glas & en kopp & internationell/t/a & middag & i dag/idag \\
        problem & fixa/fixar & säng/en/ar/arna & kök & snäll/t/a & en påse \\
        kylskåp & koka/kokar & en potatis & hacka/hackar & lök & varandra \\
        laga/r & mat(en) & mexikansk/t/a & burk/en/ar/arna & fisk/en/ar/arna & öppna/r \\
        fy & surströmming & lukta/r & illa & nog & baka/r \\
        bröd/et/\_/en & soppa/n/soppor/na & chili/n & undra/r & tro/r & rulla/r \\
        dukad & person(en)(er)(erna) & olik/t/a & sallad/en/er/erna & god/tt/a & smaka(r) \\
        efteråt &  &  &  &  &  \\
        \hline
    \end{tabular}
\end{center}

\begin{questions}
    \begin{multicols}{2}
        \raggedcolumns
        \question fork \f[gaffel,gaffeln,gafflar,gafflarna]
        \question plate \f[tallrik,tallriken,tallrikar,tallrikarna]
        \question knife \f[kniv,kniven,knivar,knivarna]
        \question spoon \f[sked,skeden,skedar,skedarna]
        \question napkin \f[servett,servetten,servetter,servetterna]
        \question deep \f[djup/djupt/djupa]
        \question bowl \f[djup tallrik]
        \question glass \f[glas,glaset,glas,glasen]
        \question cup \f[kopp,koppen,koppar,kopparna]
        \question international \f[internationell/t/a]
        \question today \f[i dag/idag]
        \question dinner, noon \f[middag,middagen,middagar,middagarna]
        \question problem \f[problem,problemet,problem,problemen]
        \question fix/arrange \f[fixa/fixar]
        \question bed \f[säng/en/er/erna]
        \question kitchen \f[kök(et)()(en)]
        \question kind, please \f[snäll/t/a]
        \question bag \f[påse,påsen,påsar,påsarna]
        \question refrigerator \f[kylskåp,kylskåpet,kylskåp,kylskåpen]
        \question boil \f[koka,kokar]
        \question potato \f[potatis,potatisen,potatisar,potatisarna]
        \question chop \f[hacka/hackar]
        \question onion \f[lök,löken,lökar,lökarna]
        \question each other \f[varandra]
        \question prepare \f[laga,lagar]
        \question food \f[mat,maten]
        \question Mexican \f[mexikansk/t/a]
        \question can \f[burk/en/ar/arna]
        \question fish \f[fisk/en/ar/arna]
        \question open \f[öppna/r]
        \question ugh, phew \f[fy]
        \question sour herring \f[surströmming/en/ar/arna]
        \question smell \f[lukta/r]
        \question bad \f[illa]
        \question probably \f[nog]
        \question bake \f[baka/r]
        \question bread \f[bröd/et/\_/en]
        \question soup \f[soppa/n/soppor/sopporna]
        \question chili pepper \f[chili/n]
        \question wonder \f[undra/r]
        \question believe, think \f[tro/r]
        \question roll \f[rulla/r]
        \question served(table) \f[duka(d)(t)(de)]
        \question person \f[person(en)(er)(erna)]
        \question different \f[olik/t/a]
        \question salad \f[sallad(en)(er)(erna)]
        \question good \f[god,gott,goda]
        \question taste \f[smaka(r)]
        \question afterwards \f[efteråt]
    \end{multicols}
\end{questions}

    \uplevel{\bfseries Fill in the correct Swedish word (EngToSwe1, Lektion 13)}

\begin{flushleft}
    \textbf{Time of the years}
    \begin{itemize}
        \item ett år
        \item fyra årstider
        \item tolv månader
    \end{itemize}

    \textbf{Questions}
    \begin{itemize}
        \item Hur, när, var, varför, vem, vilken $+$ reverse main clause
        \begin{itemize}
            \item Hur reser du till Stockholm? $\Rightarrow$ Jag tar tåget.
        \end{itemize}
        \item Tar du tåget till Stockholm? $\Rightarrow$ $\left\{
            \begin{array}{l}
                \text{Ja, det gör jag.} \\
                \text{Nej, det gör jag inte.}
            \end{array}
        \right.$
        \item väl
        \begin{itemize}
            \item Du tar väl tåget till Stockholm? $\Rightarrow$ $\left\{
                \begin{array}{l}
                    \text{Ja, det gör jag.} \\
                    \text{Nej, det gör jag inte.}
                \end{array}
            \right.$
            \item Du tar väl \textbf{inte} tåget till Stockholm? $\Rightarrow$ $\left\{
                \begin{array}{l}
                    \text{\textbf{Jo}, det gör jag.} \\
                    \text{Nej, det gör jag inte.}
                \end{array}
            \right.$
        \end{itemize}
    \end{itemize}
\end{flushleft}

\begin{center}
    \begin{tabular}{|c c c c c c|}
        \hline
        januari & februari & mars & april & maj & juni \\
        juli & augusti & september & oktober & november & december \\
        vinter & vår & sommar & höst & årstid & månad \\
        slut &  &  &  &  &  \\
        \hline
    \end{tabular}
\end{center}

\begin{questions}
    \begin{multicols}{2}
        \raggedcolumns
        \question January \f[januari]
        \question February \f[]
        \question March \f[]
        \question April \f[]
        \question Maj \f[]
        \question June \f[]
        \question July \f[]
        \question August \f[]
        \question September \f[]
        \question October \f[]
        \question November \f[]
        \question December \f[]
        \question winter \f[vinter,vintern,vintrar,vintrarna]
        \question spring \f[vår(en)(ar)(arna)]
        \question summer \f[sommar,sommaren,somrar,somrarna]
        \question fall \f[höst(en)(ar)(arna)]
        \question season \f[årstid(en)(er)(erna)]
        \question month \f[månad(en)(er)(erna)]
        \question end \f[slut/et/\_/en]
    \end{multicols}
\end{questions}


    \clearpage
    \bibliographystyle{plain}
    \bibliography{swe_texts}

\end{document}