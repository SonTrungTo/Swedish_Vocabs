\documentclass[addpoints,a3paper,11pt]{exam}
\usepackage{natbib}
\usepackage{amsmath}
\usepackage[swedish]{babel}
\usepackage[affil-it]{authblk}
\usepackage[utf8]{inputenc}
\usepackage{multicol,microtype}
\newcommand{\f}[1][{}]{\fillin[#1][0.5in]}
\newcommand{\note}[1]{\textit{#1}}
\pagestyle{headandfoot}
\chead[Leadoo Marketing Technologies \\
    Swedish self-test vocabulary \\
    Son To]{yes}
\cfoot{Page \thepage\ of \numpages}
\rfoot{\iflastpage{THE END}{Please go on to the next page\ldots}}
\begin{document}
    \title{\bfseries Swedish Master}
    \author{Son To}
    \affil{Leadoo Marketing Technologies
    \thanks{I thank my employer!}}
    \date{25. heinäkuuta 2021}
    \maketitle

    \begin{coverpages}
        This is the review of all vocabularies that appear
        in various texts,
        namely \cite{EngToSwe1},
        in my attempt to master Swedish, once and for all.
    \end{coverpages}

    \begin{center}
        \fbox{\fbox{\parbox{5.5in}{\centering
            This is the self-review made by me using
            the language \TeX\ written by Donald Knuth with
            \LaTeX\ as its successor in implementation. The
            documentation for this review comes from exam.cls
            made by Philip Hirschhorn.}}}
    \end{center}

    \vspace{0.1in}

    \makebox[\textwidth]{Name:\enspace\hrulefill}

    \vspace{0.2in}

    \makebox[\textwidth]{Year of birth:\enspace\hrulefill}

    \uplevel{\bfseries Fill in the correct Swedish word (EngToSwe1, pronounciation)}
\begin{center}
    \begin{tabular}{|c c c c c c|}
        \hline
        en arm & en hand & en katt & ett glas & ett par & en tomat \\
        en boll & en doktor & en klocka & ett kilo & en orm & en ost \\
        \hline
    \end{tabular}
\end{center}

\begin{questions}
    \begin{multicols}{3}
        \raggedcolumns
        \question an arm \fillin
        \question a hand \fillin
        \question a cat \fillin
        \question a ball \fillin
        \question a doctor \fillin
        \question a clock \fillin
        \question a kilogram \fillin
        \question a snake \fillin
        \question a cheese \fillin
    \end{multicols}
\end{questions}

    \clearpage
    \bibliographystyle{plain}
    \bibliography{swe_texts}

\end{document}